%!TEX root = ../main.tex
\chapter{Calcolo dei gruppi di Galois}
%%%%%%%%%%%%%%%%%%%%%%%%%%%%%%%%%%%%%%%%%%%
%
%LEZIONE 22/11/2016 - SETTIMA SETTIMANA (2)
%
%%%%%%%%%%%%%%%%%%%%%%%%%%%%%%%%%%%%%%%%%%%
%%%%%%%%%%%%%%%%%%%
%CAMPI CICLOTOMICI%
%%%%%%%%%%%%%%%%%%%
\section{Campi ciclotomici}

In questo paragrafo studieremo il gruppo di Galois di \(\Q[\z_m]\), dimostrando in particolare che
\[
	U(\Z/m\Z) \cong \Gal(\Q[\z_m]:\Q) \qquad\text{tramite}\qquad a\mapsto \s_a(\z_m)=z_m^a.
\]
Infine studieremo alcuni casi particolari.

Come prima cosa elenchiamo alcune proprietà, molte delle quali già note, riguardo a \(\Q[\z_m]\) che saranno utili alla nostra tesi.

Da ora in avanti faremo uso di queste notazioni:
\[
	\z = \z_m = e^{i\,\frac{2\p}{m}} \qquad\text{e}\qquad G = \Gal(\Q[\z]/\Q).
\]

\begin{pr}\label{campiCiclo1}
	\(\Q[\z]/\Q\) è un'estensione di Galois.
\end{pr}

\begin{proof}
	Per la \hyperref[th:caratterizzazioneEstensioniGalois]{caratterizzazione delle estensioni di Galois}, infatti \(\Q[\z]\) è il campo di spezzamento di
	\[
		X^m-1 \in \Q[X].\qedhere
	\]
\end{proof}

\begin{pr}\label{campiCiclo2}
	\(X^m-1\) è separabile.
\end{pr}

\begin{proof}
	Sia \(f(X)=X^m-1\). Sappiamo dalla \hyperref[pr:caratterizzazioneRadiciMultiple]{caratterizzazione delle radici multiple} che se \(f\) non fosse semplice si avrebbe
	\[
		(f,f')\neq 1.
	\]
	D'altronde \(f'(X)=n\,X^{n-1}\) e \(0\) non è una radice di \(f\), quindi \((f,f')=1\).
\end{proof}

\begin{pr}\label{campiCiclo3}
	La mappa
	\[
		U(\Z/m\Z) \longrightarrow \Gal(\Q[\z]/\Q), k \longmapsto \s_k\colon \z \mapsto \z^k
	\]
	è un omomorfismo iniettivo di gruppi.
\end{pr}

\begin{proof}
	Mostriamo che è un omomorfismo: siano \(a,b\in U(\Z/m\Z)\), avremo
	\[
		\s_a\circ \s_b(\z) = \s_a(\z^b) = \z^{b\cdot a} = \s_{a\,b}(\z).
	\]
	Quindi le operazioni vengono conservate. Per cui, dal momento che \(k\mapsto \s_k\) è ben definito, abbiamo un omomorfismo di gruppi.
	
	Per dimostrare che è iniettivo, mostriamo che il nucleo è banale:
	\[
		\s_k(\z) = \z \iff \z^k = \z \iff k=1
	\]
	in quanto \(k\in U(\Z/m\Z)\) implica \((k,m)=1\). Quindi l'omomorfismo è iniettivo.
\end{proof}

\begin{oss}
	Per dimostrare che è suriettivo, ho bisogno di dimostrare che
	\[
		\Gal(\Q[\z]/\Q) = \j(m).
	\]
\end{oss}

\begin{pr}\label{campiCiclo4}
	Vale la seguente identità:
	\[
		X^m-1 = \prod_{d\mid m} \Phi_d(X) \qquad\text{dove}\qquad \Phi_d(X) = \prod_{\substack{k=1\\(k,d)=1}}^d (X-\z_d^m).
	\]
\end{pr}

\begin{proof}
	Sappiamo che le radici di \(X^m-1\) sono le \(\z^k\) con \(k=1,\ldots,m\). Quindi
	\[
		X^m-1 = \prod_{k=1}^m (X-\z^k) = \prod_{d\mid m}\prod_{\substack{k=1\\(k,m)=d}}^m (X-\z^k) = \prod_{d\mid m} \Phi_{\frac{m}{d}}(X),
	\]
	dove
	\[
		\Phi_{\frac{m}{d}}(X) = \prod_{\substack{k=1\\(k,m)=d}}^m (X-\z^k) = \prod_{\substack{k=1\\\left( \frac{k}{d},\frac{m}{d} \right)=1}}^m (X-\z^k) = \prod_{\substack{j=1\\(j,m/d)=1}}^{m/d} (X-\z^{j\,d}).\graffito{posto \(j=k/d\)}
	\]
	Ora se
	\[
		\z = \z_m = e^{i\,\frac{2\p}{m}} \implies \z^d = e^{i\,\frac{2\p}{m}d} = \z_{\frac{m}{d}}.
	\]
	Quindi
	\[
		\Phi_{\frac{n}{d}}(X) = \prod_{\substack{j=1\\(j,m/d)=1}}^{m/d}(X-\z_d^j).
	\]
	Inoltre, poiché \(d\mid m \implies m/d \mid m\), avremo
	\[
		X^m-1 = \prod_{d\mid m} \Phi_{\frac{m}{d}}(X) = \prod_{d\mid m} \Phi_d (X),
	\]
	dove
	\[
		\Phi_d(X) = \prod_{\substack{j=1\\(j,d)=1}}^d (X-\z_d^j).\qedhere
	\]
\end{proof}

\begin{pr}\label{campiCiclo5}
	Il polinomio \(\Phi_m(X)\) è a coefficienti interi ed è irriducibile.
\end{pr}

\begin{proof}
	Per mostrare che \(\Phi_m(X)\in\Z[X]\), è sufficiente mostrare che ha coefficienti razionali. Infatti, dalla proprietà precedente,
	\[
		X^m-1 = \prod_{d\mid m} \Phi_d(X),
	\]
	dove \(X^m-1\in \Z[X]\) monico, quindi per la \autoref{pr:fattoriMoniciPolinomioMonico} ogni suo fattore a coefficienti razionali è a coefficienti interi.
	Mostriamolo per induzione:
	\begin{itemize}
		\item \(\Phi_1(X)=X-1\) ha coefficienti interi.
		\item Assumiamo che \(\Phi_d(X)\in\Q[X]\) per ogni \(d<n\), segue
		      \[
			      \Phi_m(X) = \frac{X^m-1}{\prod_{\substack{d\mid m\\d<m}}^m \Phi_d(X)}\in\Q[X]
		      \]
		      in quanto rapporto di polinomi a coefficienti razionali.
	\end{itemize}
	Mostriamo ora che è irriducibile. Per definizione
	\[
		\Phi_m(X) = \prod_{\substack{k=1\\(k,m)=1}}^m (X-\z^m)
	\]
	noi vorremmo dimostrare che \(\Phi_m = f_\z\). Osserviamo che
	\[
		f_\z(X) = \prod_{\s \in G}\big(X-\s(\z)\big).
	\]
	D'altronde
	\[
		\s(\z) = \z^k \qquad\text{e}\qquad \s^{-1}(\z) = z^{k'},
	\]
	quindi \(\z = \s\circ \s^{-1}(\z)=\z^{k\,k'}\), da cui \(k\,k'\equiv_m 1\). Ovvero
	\[
		(k,m) = 1.
	\]
	Questo significa che le radici di \(f_\z\) sono della forma \(\z^k\) con \((k,m)=1\), da cui
	\[
		f_\z \mid \Phi_m.
	\]
	Resta da mostrare il viceversa. Per farlo possiamo verificare che se \((k,m)=1\) allora
	\[
		f_\z(\a) = 0 \implies f_\z(\a^k) = 0.
	\]
	In tal caso tutte le radici di \(\Phi_m\) sarebbero radici di \(f_\z\), che implicherebbe \(\Phi_m \mid f_\z\).
	Noi dimostreremo che per ogni \(p\) primo tale che \((p,m)=1\) si ha
	\[
		f_\z(\a) = 0 \implies f_\z(\a^p)=0,
	\]
	da cui, preso \(k=p_1 \cdot\ldots\cdot p_s\), avremo
	\[
		f_\z(\a) = 0 \implies f_\z(\a^{p_1}) = 0 \implies f_\z(\a^{p_1 p_2}) = 0 \implies \ldots \implies f(\a^k) = 0.
	\]
	Sia \(f_\z(\a)=0\), supponiamo per assurdo che \(p\) sia un primo tale che \(f(\a^p)\neq 0\). Da \(f_\z\mid \Phi_m\) segue \(\Phi_m(X) = f_\z(X)g(X)\). Inoltre per la definizione di \(\Phi_m\) si ha necessariamente
	\[
		\Phi_m(\a) = 0 \implies \Phi_m(\a^p)=0.
	\]
	Quindi
	\[
		0 = \Phi_m(\a^p) = f_\z(\a^p)g(\a^p) \implies g(\a^p) = 0.
	\]
	Ciò significa che \(f_\z\) ha una radice in comune con \(g(X^p)\), da cui
	\[
		\big(f_\z(X),g(X^p)\big) \neq 1.
	\]
	Se prendiamo le classi di equivalenza modulo \(p\), \(\chius{f_\z(X)},\chius{g(X^p)}\in\F_p[X]\), allora
	\[
		\big(\chius{f_\z(X)},\chius{g(X^p)}\big) \neq 1.
	\]
	Ma, modulo \(p\), \(\chius{g(X^p)}=\big(\chius{g(X)}\big)^p\). Quindi \(f_\z(X)\) e \(g(X)\) hanno una radice in comune modulo \(p\). Da
	\[
		f_\z(X) g(X) = \Phi_m(X)
	\]
	segue che \(\Phi_m(X)\) ha una radice doppia modulo \(p\). In particolare \(\Phi_m(X)\) è un fattore di \(X^m-1\), quindi anche quest'ultimo avrà una radice doppia modulo \(p\). Ciò è assurdo poiché abbiamo visto nella \autoref{campiCiclo2} che \(X^m-1\) è separabile. Quindi \(\Phi_m(X)\) è irriducibile
\end{proof}

\begin{pr}\label{campiCiclo6}
	Il grado di \(\Q[\z]/\Q\) è \(\j(m)\).
\end{pr}

\begin{proof}
	In quanto estensione algebrica semplice, avremo
	\[
		\big[\Q[\z]:\Q\big] = \deg f_\z.
	\]
	Nella proprietà precedente abbiamo dimostrato che \(f_\z = \Phi_m\). Per definizione
	\[
		\Phi_m(X) = \prod_{\substack{k=1\\(k,m)=1}}^m (X-\z^k).
	\]
	Quindi è chiaro che \(\deg \Phi_m = \j(m)\), da cui la tesi.
\end{proof}

\begin{teor}{Gruppo di Galois dei campi ciclotomici}{gruppoGaloisCiclotomici}\index{Gruppo di Galois!dei campi ciclotomici}
	Il gruppo di Galois dei campi ciclotomici \(\Gal\big(\Q[\z_m]/\Q\big)\) è isomorfo a \(U(\Z/m\Z)\), tramite
	\[
		U(\Z/m\Z) \longrightarrow \Gal(\Q[\z]/\Q), k \longmapsto \s_k\colon \z \mapsto \z^k.
	\]
\end{teor}

\begin{proof}
	Sappiamo dalla \autoref{campiCiclo3} che la mappa dell'ipotesi, è un omomorfismo iniettivo fra \(U(\Z/m\Z)\) e \(\Gal\big(\Q[\z_m]/\z\big)\). Inoltre
	\[
		\#\Gal\big(\Q[\z_m]/\Q\big) = \big[\Q[\z_m]:\Q\big],
	\]
	dove \(\big[\Q[\z_m]:\Q\big]=\j(m)\) per la proprietà precedente.
	Quindi \(U(\Z/m\Z)\) e \(\Gal\big(\Q[\z_m]/\z\big)\) hanno lo stesso numero di elementi. Ne segue che l'omomorfismo iniettivo è necessariamente un isomorfismo.
\end{proof}

\begin{oss}
	Riepiloghiamo quanto visto finora. Posto \(K_n = \Q[\z], \z = e^{\frac{2\p\,i}{n}}\), \(K_n\) è il campo di spezzamento di \(X^n-1\). Quindi \(K_n\) è Galois poiché \(X^n-1\) è separabile.
	
	\'E un'estensione abeliana, infatti
	\[
		\Gal(K_n/\Q) \cong U(\Z/n\Z) = \Set{k\in\{1,\ldots,n\} | (k,n)=1}.
	\]
	Inoltre
	\[
		[K_n:\Q] = \#\Gal(K_n/\Q) = \j(n).
	\]
	In generale se \(n=p_1^{\a_1}\cdot\ldots\cdot p_s^{\a_s}\), si ha il seguente reticolo:
	\[
		\begin{tikzcd}
			&                     & \Q[\z_n] \arrow[dll, dash] \arrow[dl, dash] \arrow[dr, dash] \arrow[drr, dash]\\
			\Q(\z_{p_1^{\a_1}}) & \ldots &       & \ldots & \Q(\z_{p_s^{\a_s}})\\
			&                     & \Q \arrow[ull, dash, "\j(p_1^{\a_1})"] \arrow[ul, dash] \arrow[ur, dash] \arrow[urr, dash, swap, "\j(p_s^{\a_s})"] \arrow[uu, dash, swap, "\j(n)"]
		\end{tikzcd}
	\]
\end{oss}

\begin{exeN}
	Si determini il campo di Galois di \(\Q[\z_8]/\Q\) e il corrispondente reticolo dei sottocampi.
\end{exeN}

\begin{sol}
	Osserviamo che
	\[
		\z_8 = \cos \frac{\p}{4} + i\,\sin \frac{\p}{4} = \frac{\sqrt{2}}{2} (1+i).
	\]
	Inoltre \(\z_8^2 = \z_4 = i\), quindi si mostra facilmente che
	\[
		\Q[z_8] = \Q[\sqrt{2},i].
	\]
	Possiamo quindi scrivere gli elementi di \(\Gal\big(\Q[\z_8]/\Q\big)\) tramite le immagini dei generatori:
	\[
		\s_0 = id; \quad \s_1=
		\left(\begin{aligned}
				\sqrt{2} & \mapsto -\sqrt{2} \\
				i        & \mapsto i
			\end{aligned}\right);\quad
		\s_2=
		\left(\begin{aligned}
				\sqrt{2} & \mapsto \sqrt{2} \\
				i        & \mapsto -i
			\end{aligned}\right);\quad
		\s_1\circ \s_2=
		\left(\begin{aligned}
				\sqrt{2} & \mapsto -\sqrt{2} \\
				i        & \mapsto -i
			\end{aligned}\right).
	\]
	D'altronde, grazie alla teoria sviluppata in questo paragrafo, abbiamo un altro modo per esprimere questi elementi.
	Sappiamo infatti che
	\[
		\Gal\big(\Q[\z_8]/\Q\big) \cong U(\Z/8\Z) = \{\pm 1,\pm 3\},
	\]
	che inducono i seguenti automorfismi:
	\[
		\t_0\colon \z_8 \mapsto \z_8; \qquad \t_1\colon \z_8 \mapsto \z_8^{-1}; \qquad \t_2\colon \z_8 \mapsto \z_8^3; \qquad \t_1\circ \t_2\colon \z_8 \mapsto \z_8^{-3}.
	\]
	Cerchiamo di dare una corrispondenza fra le due scritture:
	\begin{itemize}
		\item chiaramente \(\s_0=\t_0=id\).
		\item \(\t_1\colon \z_8 \mapsto \z_8^{-1}\), dove
		      \[
			      \z_8^{-1} = \frac{\sqrt{2}}{2} - i \frac{\sqrt{2}}{2} \implies \frac{\sqrt{2}}{2}(1+i) \mapsto \frac{\sqrt{2}}{2}(1-i).
		      \]
		      Quindi \(\t_1=\s_2\).
		\item \(\t_2\colon \z_8 \mapsto \z_8^3\), dove
		      \[
			      \z_8^3 = -\frac{\sqrt{2}}{2}+i \frac{\sqrt{2}}{2} \implies \frac{\sqrt{2}}{2}(1+i) \mapsto \frac{\sqrt{2}}{2}(-1+i).
		      \]
		      Quindi \(\t_2=\s_1\).
		\item Dai punti precedenti segue immediatamente che \(\t_1\circ \t_2 =\s_1 \circ \s_2\).
	\end{itemize}
\end{sol}

\begin{exeN}
	Si determini il gruppo di Galois di \(K_7 = \Q[\z_7]\).
\end{exeN}

\begin{sol}
	Sappiamo
	\[
		\Gal(K_7/\Q) \cong U(\Z/7\Z) = \{1,2,3,4,5,6\} \cong \Z/6\Z = \{0,1,2,3,4,5\}.
	\]
	Osserviamo che in generale è più comodo lavorare con \(U(\Z/7\Z)\) in quanto costituisce un gruppo moltiplicativo, tale struttura si avvicina di più a quella del gruppo di automorfisimi.
	
	I generatori di \(U(\Z/7\Z)\) e \(\Z/6\Z\) sono rispettivamente \(3\) e \(1\). Quindi l'isomorfismo fra i due gruppi si costruisce con la mappa \(1 \mapsto 3\).
	
	Ricordiamo dalla teoria dei gruppi che i sottogruppi di un gruppo ciclico sono in corrispondenza biunivoca con i divisori dell'ordine. In \(\Z/6\Z\) i divisori dell'ordine sono \(6,3,2,1\) a cui corrispondono
	\[
		\langle 1 \rangle = \Z/6\Z; \qquad \langle 2 \rangle; \qquad \langle 3 \rangle; \qquad \langle 0 \rangle.
	\]
	I corrispondenti del gruppo di Galois, in vista dell'isomorfismo
	\[
		\Z/6\Z \longrightarrow U(\Z/7\Z)\colon
		\begin{aligned}
			1 & \longmapsto 3  \\
			2 & \longmapsto 2  \\
			3 & \longmapsto 6  \\
			4 & \longmapsto 4  \\
			5 & \longmapsto 5  \\
			0 & \longmapsto 1
		\end{aligned}
	\]
	sono determinati da
	\begin{align*}
		\langle 1 \rangle \longleftrightarrow \langle \s_3 \rangle & = \Set{\s_{3^k} : K_7 \to K_7, \z_7 \mapsto \z_7^{3^k}} = \Gal(K_7/\Q); \\
		\langle 2 \rangle \longleftrightarrow \langle \s_2 \rangle & = \{\s_2,\s_4,\s_1=id\};                                                \\
		\langle 3 \rangle \longleftrightarrow \langle \s_6 \rangle & = \{\s_6,\s_1=id\};                                                     \\
		\langle 0 \rangle \longleftrightarrow \langle \s_1 \rangle & = \{\s_1 = id\}.
	\end{align*}
	Per il \hyperref[th:TFCG]{Teorema Fondamentale della Corrispondenza di Galois}, avremo una corrispondenza con i sottocampi di \(K_7\). I corrispondenti banali sono
	\begin{align*}
		\langle \s_3 \rangle = G      & \longleftrightarrow K_7^{\Gal(K_7/\Q)} = \Q; \\
		\langle \s_1 \rangle = \{id\} & \longleftrightarrow K_7^{\{id\}} = K_7.
	\end{align*}
	Consideriamo ora il corrispondente di \(\langle \s_6 \rangle\):
	\[
		\langle \s_6 \rangle = \langle \s_{-1} \rangle \longleftrightarrow K_7^{\langle \s_{-1} \rangle} = \Set{\a \in K_7 | \s_{-1}(\a) = \a} = \Set{\a \in K_7 | \conj{\a}=\a} = K_7 \cap \R.
	\]
	Abbiamo già visto in esempi precedenti che \(K_7 \cap \R = \Q[\cos \rfrac{2\p}{7}]\).
	Infine
	\[
		\langle \s_2 \rangle \longleftrightarrow K_7^{\langle \s_2 \rangle} = \Set{\a \in K_7 | \s_2(\a) = \a}.
	\]
	Cerchiamo di determinare esplicitamente \(K_7^{\langle \s_2 \rangle}\). Per prima cosa osserviamo che \(K_7^{\langle \s_2 \rangle}/\Q\) è Galois in quanto \(\langle \s_2 \rangle \triangleleft G\) e ciò è sempre vero nei campi ciclotomici poiché \(G\) è abeliano. Inoltre
	\[
		[K_7^{\langle \s_2 \rangle}:\Q] = \frac{[K_7:\Q]}{[K_7:K_7^{\langle \s_2 \rangle}]} = \frac{6}{\#\langle \s_2 \rangle} = 2,
	\]
	quindi \(K_7^{\langle \s_2 \rangle}\) è un'estensione quadratica di \(\Q\). Definiamo
	\[
		\h = \sum_{\s \in \langle \s_2 \rangle}\s(\z_7) = \s_2(\z) + \s_4(\z) + \s_8(\z) = \z^2+\z^4+\z.
	\]
	Osserviamo che \(\s_2(\h)=\h\), quindi \(\Q[\h] \subseteq K_7^{\langle \s_2 \rangle}\). Ora
	\[
		\Q \subseteq \Q[\h] \subseteq K_7^{\langle \s_2 \rangle} \qquad\text{dove }[K_7^{\langle \s_2 \rangle}:\Q]=2.
	\]
	Quindi se dimostriamo che \(\Q[\h]\neq \Q\) segue necessariamente \(\Q[\h]=K_7^{\langle \s_2 \rangle}\).
	Per prima cosa troviamo il polinomio minimo di \(\h\) sfruttando la \autoref{pr:polMinimoTFCG}:
	\[
		\h^G = \big\{\h,\s_2(\h),\s_3(\h),\s_4(\h),\s_5(\h),\s_6(\h)\big\} = \{\h,\s_3(\h)\}= \{\z+\z^2+\z^4,\z^3+\z^5+\z^6\}.
	\]
	Quindi
	\[
		f_\h(X) = \big(X-\h\big)\big(X-\s_3(\h)\big) = X^2 - \big(\h+\s_3(\h)\big)X+ \h\,\s_3(\h),
	\]
	dove
	\begin{align*}
		\h+\s_3(\h)  & = \z+\z^2+\z^3+\z^4+\z^5+\z^6=-1;                                          \\
		\h\,\s_3(\h) & = (\z+\z^2+\z^4)(\z^3+\z^5+\z^6) = \z^4+\z^6+1+\z^5+1+\z+1+\z^2+\z^3 = 2.
	\end{align*}
	Per cui
	\[
		f_\h(X) = x^2+x+2 \implies \h,\s_3(\h) = -\frac{1}{2}\pm \frac{1}{2}\sqrt{-7}.
	\]
	Concludendo \(Q[\h]=\Q[\sqrt{-7}]\neq \Q\), quindi \(Q[K_7^{\langle \s_2 \rangle}] = Q[\sqrt{-7}]\).
\end{sol}

\begin{oss}
	In generale se consideriamo \(\Q[\z_p]\) con \(p\ge 3\), avremo sempre che
	\[
		\Gal\big(\Q[\z_p]/\Q\big) \cong U(\Z/p\Z) = \F_p^*
	\]
	che è ciclico. Quindi ha un sottogruppo per ogni divisore dell'ordine. Per il TFCG i sottocampi sono in corrispondenza biunivoca con i divisori di \(\j(p)= p-1\).
	
	In particolare avremo sempre
	\[
		\Q\left[\cos \frac{2\p}{p}\right]\text{ di grado }\frac{p-1}{2} \qquad\text{e}\qquad \Q\left[\sqrt{(-1)^{\frac{p-1}{2}}p}\right]\text{ di grado }2.
	\]
\end{oss}
%%%%%%%%%%%%%%%%%%%%%%%%%%%%%%%%%%%%%%%%%%
%
%LEZIONE 28/11/2016 - OTTAVA SETTIMANA (1)
%
%%%%%%%%%%%%%%%%%%%%%%%%%%%%%%%%%%%%%%%%%%
\begin{pr}
	Sia \(p\) primo, allora vale la seguente identità
	\[
		D_{\Phi_p} = -p^{p-2}.
	\]
\end{pr}

\begin{proof}
	Per definizione di discriminante
	\[
		D_f = \prod_{i<j} (\a_i-\a_j)^2 \qquad\text{dove \(\a_i,\a_j\) sono radici di \(f\)}.
	\]
	\'E possibile dimostrare la seguente definizione equivalente:
	\[
		D_f = \prod_{i<j} (\a_i-\a_j)^2 = (-1)^{\frac{n(n-1)}{2}} \prod_{j=1}^n f'(\a_j).
	\]
	Applicandola al discriminante di \(\Phi_p(X)\), otteniamo
	\[
		D_{\Phi_p} = (-1)^{\frac{\cancel{p}(p-1)}{2}} \prod_{k=1}^{p-1} \Phi_p'(\z^k) = (-1)^{\frac{p-1}{2}} \prod_{k=1}^{p-1} \Phi_p'(\z^k)\graffito{la \(p\) come esponente di \((-1)\) è ininfluente}.
	\]
	Ora
	\[
		\Phi_p'(X) = \frac{p\,X^{p-1}(X-1)-(X^p-1)}{(X-1)^2} \implies \Phi_p'(\z^k) = p\,{(\z^{p-1})}^k{(\z^k-1)}^{-1}.
	\]
	Quindi
	\[
		\begin{split}
			\prod_{k=1}^{p-1}\Phi_p'(\z^k) & = p^{p-1}{(\z^{p-1})}^{\sum_{k=1}^{p-1}k} \left( \prod_{k=1}^{p-1} (1-\z^k) \right)^{-1}(-1)^{\sum_{k=1}^{p-1}k}\\
			& = p^{p-1} \underbrace{{(\z^{p-1})}^{\frac{p(p-1)}{2}}}_{=1} \underbrace{\Phi_p(1)^{-1}}_{=p^{-1}} (-1)^{\frac{\cancel{p}(p-1)}{2}}\\
			& = (-1)^{\frac{p-1}{2}}p^{p-2}.
		\end{split}
	\]
	Da cui
	\[
		D_{\Phi_p} = (-1)^{\frac{p-1}{2}}(-1)^{\frac{p-1}{2}} p^{p-2}
	\]
	DA FINIRE PERCHE' SBAGLIATO
\end{proof}

\begin{prop}{Sottocampi di \(\Q[\z_p]\)}{sottocampipCiclotomico}
	Consideriamo \(\Q[\z],\z=e^{\frac{2\p\,i}{p}}\) e sia \(G=\Gal\big(\Q[\z]/\Q\big)\). Per ogni \(H\subseteq G\), sia
	\[
		\h_H = \sum_{h\in H} \z^h \in \Q[\z].
	\]
	Allora \(\Q[\z]^H = \Q[\h_H]\).
\end{prop}

\begin{proof}
	Da questo momento faremo uso dell'isomorfismo canonico di \(G\) con \((\Z/p\Z)^*\), rendendo di fatto indistinguibili di due gruppi tramite la mappa
	\[
		k \longmapsto \s_k\colon \z \mapsto \z^k.
	\]
	Per prima cosa osserviamo che
	\[
		\s_k(\h_H) = \sum_{h\in H}\z^{k\,h},\,\fa \s_k\in G.
	\]
	Inoltre, per ogni \(k\in H\) si avrà \(\s_k(\h_H)=\h_H\) da cui \(\Q[\h_H]\subseteq \Q[\z]^H\).
	Quindi, per dimostrare l'uguaglianza, basta mostrare che
	\[
		\big[\Q[\h_H]:\Q\big] = \big[\Q[\z]^H:\Q\big] \qquad\text{dove }\big[\Q[\z]^H:\Q\big] = \frac{\big[\Q[\z]:\Q\big]}{\big[\Q[\z]:\Q[\z]^H\big]} = \frac{\#G}{\#H},
	\]
	ovvero che
	\[
		\big[\Q[\h_H]:\Q\big] = \big[(\Z/p\Z)^*:H\big].
	\]
	Sfruttando l'espressione del polinomio minimo che abbiamo \hyperref[pr:polMinimoTFCG]{precedentemente dimostrato}, avremo che
	\[
		\big[\Q[\h_H]:\Q\big] = \deg f_{\h_H} = \#(\h_H)^G.
	\]
	Quindi il teorema si riduce a verificare che
	\[
		\#(\h_H)^G = [G:H].
	\]
	Se \(y\in (\Z/p\Z)^*\) definiamo il seguente periodo:
	\[
		\h_{y\,H} = \sum_{h\in H} \z^{y\,h}.
	\]
	In altre parole \(\h_{y\,H} = \s_y(\h_H)\).
	Osserviamo che, presi \(y_1,y_2\in (\Z/p\Z)^*\), se \(y_1H=y_2H\), allora chiaramente \(\h_{y_1H}=\h_{y_2H}\).
	Inoltre se \(y_1H \neq y_2H\), allora sosteniamo che
	\[
		\h_{y_1H} \neq \h_{y_2H}.
	\]
	Ricordiamo che le classi laterali costituiscono una partizione per il gruppo, quindi
	\[
		y_1H \neq \y_2H \implies y_1H \cap y_2H = \emptyset.
	\]
	Inoltre sappiamo che \((1,\z,\ldots,\z^{p-2})\) è una \(\Q\)-base di \(\Q[\z]\). D'altronde anche \((\z,\z^2,\ldots,\z^{p-1})\) lo è. Infatti
	\[
		\Q[\z] \longrightarrow \Q[\z], \a\longmapsto \z\a
	\]
	è un'applicazione lineare di \(\Q\)-spazi vettoriali invertibile. Per cui
	\[
		\h_{y_1H}-\h_{y_2H} = \sum_{h\in H} \z^{y_1H}-\sum_{h\in H}\z^{y_2H} \neq 0
	\]
	proprio perché \((\z,\z^2,\ldots,\z^{p-1})\) è una base.
	
	Infine
	\[
		(\h_H)^G = \Set{\s_k\h_h | k\in(\Z/p\Z)^*} = \Set{\h_{k\,H} | k\in (\Z/p\Z)^*} = \Set{\h_{g_1 H},\ldots,\h_{g_s H}}.
	\]
	Da cui \(\#(\h_H)^G=[G:H]\) in quanto \(G/H=\Set{g_1 H,\ldots,g_s H}\).
\end{proof}

\begin{oss}
	In particolare vale
	\[
		f_{\h_H}(X) = \prod_{j=1}^s (X-\h_{g_j H}).
	\]
\end{oss}

\begin{oss}
	Ricordiamo dalla definizione di classe laterale, che \(g_1 H = g_2 H \iff g_1 g_2^{-1}\in H\). Quindi
	\[
		\#\Set{y\in G | yH = kH} = \#\Set{k\,h | h\in H} = \#H,
	\]
	da cui
	\[
		\prod_{k\in G}(X-\h_{kH}) = \prod_{k=1}^{p-1} (X-\h_{kH}) = f_{\h_H}(X)^{\#H}.
	\]
\end{oss}

\begin{prop}{Due sottocampi importanti di \(\Q[\z_p]\)}{sottocampiImporatntipCiclotomici}
	Sia \(G=\Gal\big(\Q[\z_p]/\Q\big) = (\Z/p\Z)^*\). \(G\) è ciclico quindi \(G=\langle g \rangle\). Allora
	\begin{itemize}
		\item Il sottocampo associato a \(\langle g^{\frac{p-1}{2}} \rangle\) è \(\Q \left[ \cos \frac{2\p}{p} \right]\).
		\item Il sottocampo associato a \(\langle g^{2} \rangle\) è \(\Q \left[ \sqrt{(-1)^{\frac{p-1}{2}}p} \right]\).
	\end{itemize}
\end{prop}

\begin{proof}
	Dal momento che \(G=\langle g \rangle\) è ciclico, sappiamo che ogni sottogruppo \(H\le G\) è del tipo
	\[
		H = \langle g^d \rangle \qquad\text{con }d\mid p-1.
	\]
	Inoltre avremo \(\#H = \frac{p-1}{d}\).
	Nel nostro caso \(\langle g^{\frac{p-1}{2}} \rangle\) ha \(2\) elementi.
	Ora \(\langle g^{\frac{p-1}{2}} \rangle = \langle -1 \rangle\), quindi per la proposizione precedente, il sottocampo associato sarà
	\[
		\Q[\z_p]^{\langle -1 \rangle} = \Q[\h_{\langle -1 \rangle}]
	\]
	che sarà un'estensione di grado \(\frac{p-1}{2}\) su \(\Q\). Infatti
	\[
		\Q[\h_{\langle -1 \rangle}] = \Q[\z+\z^{-1}] = \Q\left[\cos \frac{2\p}{p}\right].
	\]
	Descriviamo ora il sottocampo associato a \(H=\langle g^2 \rangle\). Sappiamo
	\[
		\#H = \frac{p-1}{2} \qquad\text{e}\qquad \h_H = \sum_{t=1}^{\frac{p-1}{2}} \z^{g^{2t}}.
	\]
	Inoltre sappiamo che \(\deg f_{\h_{\langle g^2 \rangle}} = 2\). Ora
	\[
		\s_g (\h_{\langle g^2 \rangle}) = \h_{g\langle g^2 \rangle} = \sum_{k=1}^{\frac{p-1}{2}} \z^{g^{2k+1}}.
	\]
	Da cui, come abbiamo visto nella prima osservazione alla proposizione,
	\[
		f_{\h_{\langle g^2 \rangle}}(X) = (X-\h_{\langle g^2 \rangle}) (X-\h_{g\langle g^2 \rangle}) = X^2-(\h_{\langle g^2 \rangle}+\h_{g \langle g^2 \rangle})X + \h_{\langle g^2 \rangle}\h_{g \langle g^2 \rangle}.
	\]
	Dove
	\[
		\h_{\langle g^2 \rangle} + \h_{g \langle g^2 \rangle} = \sum_{k=1}^{\frac{p-1}{2}} \z^{g^{2k}} + \sum_{k=1}^{\frac{p-1}{2}} \z^{g^{2k+1}} = \z+\z^2 + \ldots + \z^{p-1} = -1.
	\]
	Resta da calcolare \(\h_{\langle g^2 \rangle} \h_{g\langle g^2 \rangle}\) che sappiamo essere in \(\Z\).
	Per semplicità di notazione scriviamo
	\[
		\h_0 = \h_{\langle g^2 \rangle} \qquad\text{e}\qquad \h_1 = \h_{g \langle g^2 \rangle}.
	\]
	Se riusciamo a determinare \(\h_0-\h_1=A\), avremo
	\[
		\begin{cases}
			\h_0+\h_1 = -1 \\
			\h_0-\h_1 = A
		\end{cases}
		\implies \h_0 = \frac{1}{2}(-1+A), \h_1 = \frac{1}{2}(-1-A).
	\]
	Ora
	\[
		A = \h_0-\h_1 = \sum_{k=1}^{\frac{p-1}{2}} \z^{g^{2k}} - \sum_{k=1}^{\frac{p-1}{2}} \z^{g^{2k+1}} = \sum_{j=1}^{p-1} \e_j \z^j,
	\]
	dove
	\[
		\e_j = \lege{j}{p} = 	\begin{cases}
			1  & j=g^{2k}    \\
			-1 & j=g^{2k+1}
		\end{cases}
	\]
	è il simbolo di Legendre.\graffito{guardare gli appundi di TN410 per alcune proprietà sul simbolo di Legendre}
	Tramite alcune manipolazioni algebriche che sfruttano le proprietà del simbolo di Legendre, si può dimostrare che
	\[
		A^2 = \lege{-1}{p}p.
	\]
	Da cui
	\[
		A = \pm \sqrt{(-1)^{\frac{p-1}{2}}p} \implies \h_0,\h_1 = \frac{1}{2} \left( 1\pm \sqrt{(-1)^{\frac{p-1}{2}}p} \right)
	\]
	Che ci dice proprio \(\Q[\h_0]=\Q \left[ \sqrt{(-1)^{\frac{p-1}{2}}p} \right] \).
\end{proof}
%%%%%%%%%%%%%%%%%%%%%%%%%%%%%%%%%%%
%GRUPPO TRANSITIVO DI UN POLINOMIO%
%%%%%%%%%%%%%%%%%%%%%%%%%%%%%%%%%%%
\section{Gruppo transitivo di un polinomio}

Nei prossimi paragrafi \(f(X)\in F[X]\) sarà sempre monico e separabile.

Se \(f\in F[X]\) separabile, allora il suo campo di spezzamento \(F_f\) è Galois su \(F\). In particolare se
\[
	f(X) = \prod_{j=1}^n (X-\a_j), \qquad\text{con }\a_1,\ldots,\a_n \in F_f,
\]
preso
\[
	\b = \prod_{i<j} (\a_i-\a_j),
\]
certamente \(\b\in F_f\), inoltre \(\b^2=D_f\). Quindi
\[
	F \subseteq F[\sqrt{D_f}] \subseteq F_f,
\]
dove \(F[\sqrt{D_f}]/F\) è quadratica se \(D_f\) non è un quadrato perfetto, altrimenti \(F[\sqrt{D_f}]=F\).

\begin{defn}{Gruppo di Galois di un polinomio}{gruppoGaloisPolinomio}\index{Gruppo di Galois!di un polinomio}
	Sia \(f\in F[X]\) un polinomio separabile. Definiamo il \emph{gruppo di Galois di \(f\)} come il gruppo di Galois di \(F_f/F\):
	\[
		\Gal(f) := \Gal(F_f/F).
	\]
\end{defn}

\begin{prop}{}{GaloisSottogruppoPermutazioni}
	Sia \(E/F\) un'estensione di Galois. Dove \(E=F_f\) con \(f\in F[X],n=\deg f\). Allora
	\[
		\Gal(f) \lesssim S_n.
	\]
\end{prop}

\begin{proof}
	Supponiamo che
	\[
		f(X) = (X-\a_1)(X-\a_2)\cdot\ldots\cdot(X-\a_n) \implies E=F_f = F[\a_1,\ldots,\a_n].
	\]
	Vogliamo dimostrare che
	\[
		\Gal(f) \longrightarrow \text{Sym}\big(\{\a_1,\ldots,\a_n\}\big) \cong S_n, \s \longmapsto
		\begin{pmatrix}
			\a_1     & \ldots & \a_n      \\
			\s(\a_1) & \ldots & \s(\a_n)
		\end{pmatrix}
	\]
	è un omomorfismo.
	Ma ciò segue da
	\[
		f\big(\s(\a_j)\big) = \s\big(f(\a_j)\big) = 0 \implies \,\fa j\,\exists!\, k:\s(\a_j)=\a_k.
	\]
	Da cui segue che \(\Gal(f)\lesssim S_n\) per il teorema fondamentale dell'omomorfismo di gruppi.
\end{proof}

\begin{oss}
	Da ciò segue che \(\#\Gal(f) \mid n!\). Inoltre se \(f\) è irriducibile e \(f(\a)=0\) avremo
	\[
		F \subset F[\a] \subseteq F_f \qquad\text{con }\big[F[\a]:F\big] = n.
	\]
	Da cui \(n \mid \#\Gal(f)\).
\end{oss}

\begin{ese}[Controesempio]
	Il viceversa non è sempre vero, ad esempio se prendiamo
	\[
		f(X) = (X^2-2)(X^2+1),
	\]
	avremo che \(f\) ha grado \(4\) ed è riducibile. Ma il suo campo di spezzamento \(\Q[\sqrt{2},i]\) ha ancora grado \(4\).
\end{ese}

\begin{cor}
	Se \(f\) ha grado \(n>2\), allora
	\[
		\Gal(f) \not\cong \Z/n!\Z.
	\]
\end{cor}

\begin{proof}
	Segue da \(\Z/n!\Z\not\cong S_n\), infatti pur avendo la stessa dimensione, \(\Z/n!\Z\) è abeliano mentre \(S_n\) non lo è.
\end{proof}
%%%%%%%%%%%%%%%%%%%%%%%%%%%%%%%%%%%%%%%%%%
%
%LEZIONE 30/11/2016 - OTTAVA SETTIMANA (2)
%
%%%%%%%%%%%%%%%%%%%%%%%%%%%%%%%%%%%%%%%%%%
\begin{defn}{Sottogruppo transitivo}{sottogruppoTransitivo}
	Un sottogruppo \(H\le S_n\) si dice \emph{transitivo} se
	\[
		\fa i,j \in \{1,\ldots,n\}\,\ex \s\in H: \s(i)=j.
	\]
\end{defn}

\begin{table}[tp]
	\caption{Sottogruppi transitivi di \(S_n\)}
	\centering
	\begin{tabular}{ccc}
		\toprule
		\textbf{Gruppo} & \textbf{\#Sottogruppi transitivi} & \textbf{Descrizione}    \\
		\midrule
		\(S_2\)         & \(1\)                             & \(S_2\)                 \\
		\(S_3\)         & \(2\)                             & \(S_3=D_3, A_3=C_3\)    \\
		\(S_4\)         & \(5\)                             & \(S_4,A_4,C_4,D_4,V\)   \\
		\(S_5\)         & \(5\)                             & \(S_5,A_5,C_5,D_5,F_5\) \\
		\(S_6\)         & \(16\)                                                      \\
		\(S_7\)         & \(7\)                                                       \\
		\(S_8\)         & \(50\)                                                      \\
		\(\vdots\)                                                                    \\
		\(S_{24}\)      & \(26813\)                                                   \\
		\bottomrule
	\end{tabular}
\end{table}

\begin{oss}
	Chiaramente è possibile dare una nozione più generale di sottogruppo transitivo, in questo corso si è preferito richiamarla solo per i sottogruppi di \(S_n\).
\end{oss}

\begin{ese}
	\(S_n,A_n,C_n=\langle(1\ 2\ \ldots\ n)\rangle, D_n\) sono tutti sottogruppi transitivi su \(S_n\).
	
	\(S_{n-1}\le S_n\) non è transitivo.
\end{ese}

\begin{ese}[Sottogruppi isomorfi ma diversamente transitivi]
	Consideriamo \(\langle(1\, 2),(3\ 4)\rangle\le S_4\). Osserviamo che tale sottogruppo non è transitivo (ad esempio non esiste \(\s\) tale che \(\s(2)=3\)) e che è isomorfo a \(C_2\times C_2\).
	
	Consideriamo ora il gruppo di Klein
	\[
		V = \Set{(1),(1\ 2)(3\ 4),(1\ 3)(2\ 4),(1\ 4)(2\ 3)}.
	\]
	Si mostra facilmente che \(V\) è transitivo. Inoltre anche \(V\) è isomorfo a \(C_2\times C_2\).
	Quindi la transitività non è una proprietà invariante per isomorfismi.
\end{ese}

\begin{prop}{Caratterizzazione gruppi di Galois transitivi}{caratterizzazioneGaloisTransitivi}
	Sia \(f\in F[X]\) separabile. Allora \(f\) è irriducibile se e soltanto se \(\Gal(f)\lesssim S_n\) è transitivo sulle radici.
\end{prop}

\begin{proof}
	\graffito{\(\Rightarrow)\)}Supponiamo che \(f\) sia irriducibile. Quindi avremo
	\[
		f(X) = \prod_{j=1}^n (X-\a_j) \qquad\text{e}\qquad F_f = F[\a_1,\ldots,\a_n].
	\]
	Siano \(\a,\b\in \{\a_1,\ldots,\a_n\}\). \(F[\a],F[\b]\) sono campi col gambo \(f\). Vi è chiaramente un isomorfismo \(F[\a]\xrightarrow{\sim} F[\b],\a \mapsto \b\). Definisco \(f_1\) la composizione di tale isomorfismo con l'immersione in \(F_f\):
	\[
		f_1\colon F[\a] \xrightarrow{\sim} F[\b] \hookrightarrow F_f.
	\]
	Sappiamo, assumendo \(\a=\a_1\), che \(f_1\) può essere esteso, passo dopo passo, a un \(F\)-omomorfismo\graffito{per la costruzione possiamo usare il \autoref{th:corrispondenzaFOmomorfismiEstensioniSemplici}}
	\[
		f_n\colon F[\a_1,\ldots,\a_n] = F_f \longrightarrow F_f, \a \longmapsto \b.
	\]
	Quindi ho trovato \(f_n\in \Aut(F_f/F)=\Gal(F_f/F)\) tale che \(f_n(\a)=\b\). Dal momento che posso trovare una tale mappa per ogni coppia di radici di \(f\), ciò significa che \(\Gal(f)\) è transitivo su \(S_n\).
	
	\graffito{\(\Leftarrow)\)}Supponiamo che \(\Gal(f)\) sia transitivo. Sia \(g(X)\in F[X]\) un fattore irriducibile di \(f(X)\).
	Se \(\a\) è una radice di \(g\), allora per ogni radice \(b\) di \(f\), sia \(\s\in \Gal(f)\) tale che \(\s(\a)=\b\). Da cui
	\[
		g(\b) = g\big(\s(\a)\big) = \s\big(g(\a)\big) = 0.
	\]
	Quindi ogni radice di \(f\) è radice di \(g\), ne segue che \(f\mid g\) e quindi \(f(X)=g(X)\) irriducibile.
\end{proof}

\begin{ese}
	Consideriamo \(f(X) = (X^2-2)(X^2+1)\). Sappiamo che \(\Q_f=\Q[\sqrt{2},i]\) e
	\[
		\Gal(\Q_f,\Q) = \Set{\s_1,\s_4=
			\left(\begin{aligned}
				\sqrt{2} & \mapsto \pm\sqrt{2} \\
				i        & \mapsto \pm i
			\end{aligned}\right),
			\s_2,\s_3=
			\left(\begin{aligned}
				\sqrt{2} & \mapsto \pm \sqrt{2} \\
				i        & \mapsto \mp i
			\end{aligned}\right)}
	\]
	Ora, se denotiamo \(\Set{\a_1=\sqrt{2},\a_2=-\sqrt{2},\a_3=i,\a_4=-i}\), possiamo sfruttare l'immersione isomorfa
	\[
		\Gal(f) \xhookrightarrow{\sim}\text{ Sym}\big(\{\a_1,\a_2,\a_3,\a_4\}\big) \cong S_4
	\]
	In particolare avremo
	\begin{align*}
		\left(\begin{aligned}\sqrt{2} & \mapsto \sqrt{2}\\i & \mapsto i\end{aligned}\right)  & \longleftrightarrow (1)    & \left(\begin{aligned}\sqrt{2} & \mapsto -\sqrt{2}\\i & \mapsto i\end{aligned}\right)  & \longleftrightarrow (1\ 2)        \\
		\left(\begin{aligned}\sqrt{2} & \mapsto \sqrt{2}\\i & \mapsto -i\end{aligned}\right) & \longleftrightarrow (3\ 4) & \left(\begin{aligned}\sqrt{2} & \mapsto -\sqrt{2}\\i & \mapsto -i\end{aligned}\right) & \longleftrightarrow (1\ 2)(3\ 4)
	\end{align*}
	Ovvero \(\Gal(f) \cong \Set{(1),(1\ 2),(3\ 4),(1\ 2)(3\ 4)} \le S_n\) che come ci aspettavamo dalla proposizione non è transitivo.
\end{ese}

\begin{oss}
	Se invece consideriamo \(\Q[\sqrt{2}+i]=\Q[\sqrt{2},i]\) che è il campo di spezzamento di
	\[
		g(X) = f_{\sqrt{2}+i}(X) = \prod_{j=1}^4 \big(X-\s_j(\sqrt{2}+i)\big) = X^4-2x^2+9,
	\]
	ci aspettiamo che \(\Gal(g)\) sia transitivo poiché \(g\) è irriducibile su \(\Q[X]\). Ce lo aspettiamo nonostante \(\Gal(g)\cong \Gal(f)\), poiché abbiamo visto che la transitività non è invariante per isomorfismi.
	Denotiamo \(\Set{\b_1=\sqrt{2}+i,\b_2=-\sqrt{2}+i,\b_3=\sqrt{2}-i,\b_4=-\sqrt{2}-i}\). Ora, dal momento che \(\Q[\sqrt{2}+i]=\Q[\sqrt{2},i]\), avremo che \(\Gal(g)\) ha gli stessi automorfismi di \(\Gal(f)\). In particolare
	\begin{align*}
		\s_1 & \colon \begin{aligned}\b_1 &\mapsto \b_1\\\b_2 &\mapsto \b_2\\\b_3 &\mapsto \b_3\\\b_4 &\mapsto \b_4\end{aligned} & \s_2 & \colon \begin{aligned}\b_1 &\mapsto \b_2\\\b_2 &\mapsto \b_1\\\b_3 &\mapsto \b_4\\\b_4 &\mapsto \b_1\end{aligned} & 
		\s_3 & \colon \begin{aligned}\b_1 &\mapsto \b_3\\\b_2 &\mapsto \b_4\\\b_3 &\mapsto \b_1\\\b_4 &\mapsto \b_2\end{aligned} & \s_4 & \colon \begin{aligned}\b_1 &\mapsto \b_4\\\b_2 &\mapsto \b_3\\\b_3 &\mapsto \b_2\\\b_4 &\mapsto \b_1\end{aligned}
	\end{align*}
	da cui
	\begin{align*}
		\s_1 & \longleftrightarrow (1) & \s_2 & \longleftrightarrow (1\ 2)(3\ 4) & \s_3 & \longleftrightarrow (1\ 3)(2\ 4) & \s_4 & \longleftrightarrow (1\ 4)(2\ 3).
	\end{align*}
	Ovvero \(\Gal(g) \cong V\le S_n\) che è transitivo. In conclusione \(\Gal(g)=\Gal(f)\) ed entrambi sono isomorfi a \(C_2\times C_2\le S_4\), ma solo \(\Gal(g)\) è transitivo. Questo accade poiché \(g\) è irriducibile mentre \(f\) non lo è.
\end{oss}
%%%%%%%%%%%%%%%%%%%%%%%%%%%%%%%%%%%%%%%%%%%
%GRUPPO DI UN POLINOMIO NEL GRUPPO ALTERNO%
%%%%%%%%%%%%%%%%%%%%%%%%%%%%%%%%%%%%%%%%%%%
\section{Gruppo di un polinomio nel gruppo alterno}

In questo paragrafo cercheremo di capire sotto quali ipotesi il gruppo di Galois di un polinomio \(f\) di grado \(n\), è contenuto nel gruppo alterno \(A_n\).

Cominciamo con un breve richiamo sul segno di una permutazione.

\begin{defn}{Segno permutazione}{segnoPermutazione}
	Sia \(\s\in S_n\). Scritto \(\s = c_1\circ c_2 \circ \ldots \circ c_k\) prodotto di cicli disgiunti, diremo che il \emph{segno} di \(\s\) è
	\[
		\sgn(\s) = (-1)^{l(c_1)+\ldots+l(c_k)-k},
	\]
	dove con \(l(c_j)\) indichiamo la lunghezza di \(c_j\).
\end{defn}

\begin{oss}
	Alternativamente, scritto \(\s = \t_1 \circ \ldots \circ \t_s\) prodotto di trasposizioni, potevamo definire il segno di \(\s\) come
	\[
		\sgn(\s) = (-1)^s.
	\]
	Chiaramente le due definizioni sono equivalenti.
\end{oss}

\begin{ese}
	\[
		\sgn(1\ 2) = (-1)^1 = -1 \qquad\text{e}\qquad \sgn(1\ 2\ \ldots\ n) = (-1)^{n-1}.
	\]
\end{ese}

\begin{defn}{Gruppo alterno}{gruppoAlterno}
	Considero la mappa
	\[
		\sgn\colon S_n \longrightarrow \{\pm 1\}, \s \longmapsto \sgn(\s)
	\]
	che costituisce un omomorfismo suriettivo. Definisco il \emph{gruppo alterno} di \(S_n\) come il nucleo di \(\sgn\):
	\[
		A_n:=\Ker(\sgn) = \Set{\s\in S_n | \sgn(\s) = 1}.
	\]
\end{defn}

\begin{oss}
	\(A_n\) è di fatto l'insieme delle permutazioni pari di \(S_n\). La definizione come nucleo di un omomorfismo però ci garantisce che
	\[
		A_n \trianglelefteq S_n \qquad\text{e}\qquad [S_n:A_n] = 2.
	\]
\end{oss}

\begin{ese}
	Consideriamo il gruppo delle permutazioni di ordine \(3\):
	\[
		S_3 = \Set{(1),(1\ 2),(1\ 3),(2\ 3),(1\ 2\ 3),(1\ 3\ 2)}.
	\]
	In \(S_3\) i \(2\)-cicli hanno \(\sgn=-1\) mentre i \(3\)-cicli hanno \(\sgn=1\). Quindi
	\[
		A_3 = \Set{(1),(1\ 2\ 3),(1\ 3\ 2)} \cong C_3.
	\]
\end{ese}

\begin{pr}
	Sia \(f\in F[X]\). Allora \(D_f\in F\).
\end{pr}

\begin{proof}
	Ricordiamo la definizione di discriminante
	\[
		D_f = \prod_{i<j} (\a_i-\a_j)^2.
	\]
	Tale definizione, per via del quadrato, non dipende dall'etichettatura delle radici di \(f\).
	Quindi, per ogni \(\s\in \Gal(f)\), avremo
	\[
		\s D_f = \prod_{i<j} \big(\s(\a_i)-\s(\a_j)\big)^2 = D_f
	\]
	in quanto abbiamo solo riordinato le radici. Da ciò segue
	\[
		D_f \in F_f^{\Gal(f)} = F.\qedhere
	\]
\end{proof}

\begin{defn}{Radice del discriminante}{radiceDiscriminante}
	Preso \(f\in F[X]\), definiamo
	\[
		\Delta_f = \prod_{i<j} (\a_i-\a_j) = \sqrt{D_f}.
	\]
\end{defn}

\begin{pr}
	Siano \(f\in F[X]\) e \(\s\in \Gal(f)\). Allora
	\[
		\s\Delta_f = \sgn(\s)\Delta_f.
	\]
\end{pr}

\begin{proof}
	Non fornita. La tesi è comunque intuitiva poiché \(\s\) scambia gli indici delle radici, facendo comparire un segno meno ogni volta che \(i<j\) e \(\s(i)>\s(j)\).
\end{proof}

\begin{teor}{Quando \(\Gal(f)\le A_n\)?}{GalfInAn}
	Sia \(f\in F[X]\). Allora \(\Gal(f)\le A_n\) se e soltanto se \(\Delta_f\in F\), ovvero se \(D_f\) è un quadrato perfetto in \(F\).
\end{teor}

\begin{proof}
	Dalla proprietà precedente sappiamo che \(\s\Delta_f = \sgn(\s)\Delta_f\). Da cui
	\[
		\s\in A_n \iff \s \Delta_f = \Delta_f \iff \Delta_f \in F_f^{A_n\cap \Gal(f)}.
	\]
	In particolare segue facilmente che
	\[
		F[\Delta_f] = F_f^{A_n\cap \Gal(f)}.
	\]
	Inoltre \(\big[F[\Delta_f]:F\big]\le 2\) in quanto \(\Delta_f^2=D_f\in F\).
	
	In conclusione
	\[
		\Gal(f) \le A_n \iff \Gal(f)\cap A_n = \Gal(f) \iff F[\Delta_f] = F_f^{A_n\cap \Gal(f)} = F_f^{\Gal(f)} = F.
	\]
	Da cui segue la tesi.
\end{proof}

\begin{prop}{Gruppo di Galois di un polinomio di grado \(3\)}{gruppoGaloisPolinomio3}
	Sia \(f\in F[X]\) un polinomio di grado \(3\). Allora
	\begin{itemize}
		\item Se \(f\) è irriducibile,
		      \[
			      \begin{cases}
				      \Gal(f) \cong A_3 & \text{se }D = \square     \\
				      \Gal(f) \cong S_3 & \text{se }D \neq \square
			      \end{cases}
		      \]
		\item Se \(f\) è totalmente riducibile,
		      \[
			      \Gal(f) = \{id\}.
		      \]
		\item Se \(f\) è parzialmente riducibile in un fattore di secondo grado e uno di primo,
		      \[
			      \Gal(f) \cong \Z/2\Z.
		      \]
	\end{itemize}
\end{prop}

\begin{proof}
	Per definizione \(\Gal(f)=\Gal(F_f/F)\). D'altronde sappiamo che
	\[
		\#\Gal(F_f/F) = [F_f:F] \le (\deg f)! = 6.
	\]
	Quindi \(\Gal(f)\) ha ordine un divisore di \(6\).
	
	Se \(f\) è totalmente riducibile, è chiaro che \(F_f=F\) e quindi \(\Gal(f)=\{id\}\).
	Se \(f\) ha un fattore di grado \(2\) irriducibile, allora \(F_f/F\) è un'estensione quadratica, in particolare
	\[
		\#\Gal(f) = [F_f:F] = 2 \implies \Gal(f) \cong \Z/2\Z. 
	\]
	Supponiamo ora che \(f\) sia irriducibile. Avremo che se \(f[\a]=0\), \(F \subset F[\a] \subset F_f\), da cui
	\[
		\big[F[\a]:F\big] = \deg f = 3 \implies 3 \mid [F_f:F] = \#\Gal(f).
	\]
	Per cui \(3\mid \#\Gal(f) \mid 6\), ovvero \(\#\Gal(f)\in\{3,6\}\).
	Dal teorema precedente sappiamo che \(\Gal(f)\le A_3\) se e soltanto se \(\Delta_f\in F\), ovvero se \(D_f\) è un quadrato perfetto in \(F\). Da cui
	\[
		\#\Gal(f) = [F_f:F] = 	\begin{cases}
			3 & \text{se }D_f = \square     \\
			6 & \text{se }D_f \neq \square
		\end{cases}\qedhere
	\]
\end{proof}
%%%%%%%%%%%%%%%%%%%%%%%%%%%%%%%%%%%%%%%%%%
%
%LEZIONE 14/12/2016 - DECIMA SETTIMANA (1)
%
%%%%%%%%%%%%%%%%%%%%%%%%%%%%%%%%%%%%%%%%%%
%%%%%%%%%%%%%%%%%%%%%%%%%%
%POLINOMI DI QUARTO GRADO%
%%%%%%%%%%%%%%%%%%%%%%%%%%
\section{Polinomi di quarto grado}

In questo paragrafo forniremo dei criteri per determinare il gruppo di Galois di un polinomio \(f\in F[X]\) di quarto grado che sia irriducibile e separabile.

Sappiamo che \(G_f :=\Gal(f) \subseteq \text{Sym}\{\a_1,\a_2,\a_3,\a_4\}\cong S_4\), dove \(\a_i\) sono radici di \(f\). Inoltre \(G_f\) è transitivo su \(S_4\) poiché \(f\) è irriducibile.
\(G_f\) deve pertanto essere uno dei sottogruppi di \(S_4\) elencati nella tabella \ref{tb:transitiviS4}.

\begin{table}[tp]
	\caption{Sottogruppi transitivi di \(S_4\).}
	\label{tb:transitiviS4}
	\centering
	\begin{tabular}{ccc}
		\toprule
		\textbf{Sottogruppo} & \textbf{Elementi}                                            & \textbf{Normale su \(S_4\)?} \\
		\midrule
		\(S_4\)              & \((1),(i\ j),(i\ j\ k),(i\ j\ k\ l),(i\ j)(k\ l)\)           & Sì                           \\
		\(A_4\)              & \((1),(i\ j\ k),(i\ j)(k\ l)\)                               & Sì                           \\
		\(D_4\)              & \((1),(1\ 2\ 3\ 4),(1\ 4\ 3\ 2),(1\ 3),(2\ 4),(i\ j)(k\ l)\) & No                           \\
		\(C_4\)              & \(\langle(1\ 2\ 3\ 4)\rangle\)                               & No                           \\
		\(V\)                & \((1),(1\ 2)(3\ 4),(1\ 3)(2\ 4),(1\ 4)(2\ 3)\)               & Sì                           \\
		\bottomrule
	\end{tabular}
\end{table}

\begin{oss}
	Forniamo qualche spiegazione sulla normalità dei sottogruppi transitivi di \(S_4\):
	\begin{itemize}
		\item \(S_4\) è banalmente normale in se stesso.
		\item \(A_4\) è normale in \(S_4\) poiché ha indice \(2\).
		\item \(D_4\) non è normale, infatti \((1\ 2)(1\ 2\ 3\ 4)(1\ 2) = (1\ 3\ 4\ 2)\not\in D_4\).
		\item \(C_4\) non è normale per lo stesso motivo di \(D_4\).
		\item \(V\) è normale poiché coniugando un \(k\)-ciclo si ottiene sempre un \(k\)-ciclo. In particolare tutti gli elementi di \(V\) distinti da \((1)\) sono \(2\times 2\)-cicli, quindi \(V\) è normale.
	\end{itemize}
\end{oss}

\begin{defn}{Risolvente cubica}{risolventeCubica}\index{Risolvente cubica}
	Supponiamo che \(f\in F[X]\) sia un polinomio irriducibile e separabile di grado \(4\).
	Il suo campo di spezzamento sarà \(F_f = F[\a_1,\a_2,\a_3,\a_4]\) dove \(\a_i\) sono le radici di \(f\). Presi
	\begin{align*}
		\a & = \a_1\a_2 + \a_3\a_4; & \b & = \a_1\a_3 + \a_2\a_4; & \g & = \a_1\a_4 + \a_2\a_3.
	\end{align*}
	Definiamo la \emph{risolvente cubica} di \(f\) come
	\[
		g(X) = (X-\a)(X-\b)(X-\g).
	\]
\end{defn}

\begin{oss}
	Per prima cosa osserviamo che \(F\subseteq F[\a,\b,\g]\subseteq F_f\).
	Inoltre, dal momento che \(f\) è separabile, \(\a,\b,\g\) sono tutti distinti. Ad esempio
	\[
		\a-\b = (\a_1-\a_4)(\a_2-\a_4) \neq 0.
	\]
	Infine si può facilmente verificare che \(S_4\) permuta \(\a,\b,\g\), cioè se \(\s\in S_4 = \text{Sym}\{\a_1,\a_2,\a_3,\a_4\}\), si ha
	\[
		\s\{\a,\b,\g\} = \{\a,\b,\g\}.
	\]
\end{oss}

\begin{pr}
	La risolvente cubica ha coefficienti in \(F\).
\end{pr}

\begin{proof}
	Dall'osservazione precedente sappiamo che \(S_4\) permuta \(\a,\b,\g\). In particolare da
	\[
		g(X) = (X-\a)(X-\b)(X-\g) \implies \s g(X) = g(X),\,\fa \s\in S_4.
	\]
	In particolare \(G_f \subseteq S_4\), quindi
	\[
		g(X) \in F^{G_f}[X] = F[X].
	\]
\end{proof}

\begin{prop}{Forma esplicita della risolvente cubica}{formaEsplicitaRisolventeCubica}
	Supponiamo che \(f(X)\in F[X]\) sia un polinomio separabile e irriducibile della forma
	\[
		f(X) = X^4+b\,X^3+c\,X^2+d\,X+e.
	\]
	Allora la risolvente cubica di \(f\) è
	\[
		g(X) = X^3-c\,X^2+(b\,d-4e)X+4c\,e-d^2.
	\]
\end{prop}

\begin{proof}
	Per definizione
	\[
		g(X) = (X-\a)(X-\b)(X-\g) = X^3 - (\a+\b+\g)X^2 + (\a\,\b+\a\,\g+\b\,\g)X-\a\,\b\,\g.
	\]
	Inoltre
	\[
		f(X) = \prod_{i=1}^4 (X-\a_i) = X^4-(\a_1+\a_2+\a_3+\a_4)X^3 + \ldots + \a_1\a_2\a_3\a_4.
	\]
	A questo punto è sufficiente verificare l'esattezza delle identità sui coefficienti.
\end{proof}

\begin{oss}
	Se in \(f\) sostituiamo \(X-b/4\) ad \(X\), otteniamo
	\[
		f(X-b/4) = X^4+A\,X^2+B\,X+C,
	\]
	che sappiamo avere lo stesso gruppo di Galois di \(f(X)\).
	A questo punto la risolvente cubica ha una forma più compatta:
	\[
		g(X) = X^3-A\,X^2-4C\,X+4A\,C-B^2.
	\]
\end{oss}

\begin{prop}{Campo di spezzamento della risolvente cubica}{campoSpezzamentoRisolventeCubica}
	Sia \(f\in F[X]\) un polinomio irriducibile e separabile di grado \(4\) e sia \(G_f = \Gal(f)\).
	Se \(g(X)\) è la risolvente cubica di \(f\), allora
	\[
		F_g = F_f^{V\cap G_f},
	\]
	dove \(V \le S_4\) è il gruppo di Klein.
\end{prop}

\begin{proof}
	In questa dimostrazione mostreremo solo una delle due implicazioni, poiché la seconda richiede una parte di teoria dei gruppi che esula dagli argomenti di questo corso.
	
	Per definizione sappiamo che \(F_g = F[\a,\b,\g]\). Mostriamo che \(F[\a,\b,\g]\subseteq F_f^{V\cap G_f}\).
	Se \(\s\in V\)  è facile verificare che
	\begin{align*}
		\s\a & = \a; & \s\b & = \b; & \s\g & = \g.
	\end{align*}
	In particolare ciò vale se \(\s\in V\cap G_f\), da cui
	\[
		\a,\b,\g \in F_f^{V\cap G_f} \implies F[\a,\b,\g] \subseteq F_f^{V\cap G_f}.\qedhere
	\]
\end{proof}

\begin{cor}
	L'estensione \(F[\a,\b,\g]/F\) è Galois con gruppo di Galois
	\[
		G_g \cong \frac{G_f}{V\cap G_f}.
	\]
\end{cor}

\begin{proof}
	Siccome \(V\cap G_f\) è normale in \(G_f\), avremo che \(F_f^{V\cap G_f} = F[\a,\b,\g]/F\) è normale è quindi di Galois. In particolare, sempre per la normalità del gruppo corrispondente, avremo
	\[
		G_g \cong \frac{G_f}{V\cap G_f}.\qedhere
	\]
\end{proof}

\begin{table}[tp]
	\caption{Caratterizzazione dei gruppi di Galois per polinomi di grado \(4\).}
	\label{tb:caratGaloisPolinomi4}
	\centering
	\begin{tabular}{ccc}
		\toprule
		\textbf{\(G_f\)} & \textbf{\(\#V\cap G_f\)} & \textbf{\(\#(G_f/V\cap G_f)=\#G_g=\big[F[\a,\b,\g]:F\big]\)} \\
		\midrule
		\(S_4\)          & \(4\)                    & \(6\)                                                        \\
		\(A_4\)          & \(4\)                    & \(3\)                                                        \\
		\(V\)            & \(4\)                    & \(1\)                                                        \\
		\(D_4\)          & \(4\)                    & \(2\)                                                        \\
		\(C_4\)          & \(2\)                    & \(2\)                                                        \\
		\bottomrule
	\end{tabular}
\end{table}

\begin{oss}
	Da ciò segue un'importante caratterizzazione dei gruppi di Galois dei polinomi separabili e irriducibili di grado \(4\).
	Nella tabella \ref{tb:caratGaloisPolinomi4} possiamo vedere come il gruppo di Galois \(G_g\) ci permetta di determinare \(G_f\), tranne nel caso in cui \(G_f=D_4\) o \(G_f=C_4\).
	Per dare una determinazione in quest'ultimo caso, osserviamo che se \(g(X)\) ha un fattore di grado due irriducibile, allora
	\[
		G_g = C_2 \implies F_g = F[\sqrt{D}].
	\]
	Consideriamo \(f\) in \(F[\sqrt{D}]\big[X\big]\). Se \(f\) risulta ancora irriducibile, allora
	\[
		\big[F_f:F[\sqrt{D}]\big] = 4 \implies G_f = D_4. \text{ Quindi} \qquad G_f = 	\begin{cases}
			D_4 & f\in\irr\big(F[\sqrt{D}]\big[X]\big) \\
			C_4 & \text{altrimenti}
		\end{cases}
	\]
\end{oss}

\begin{ese}
	\begin{itemize}
		Troviamo il gruppo di Galois di alcuni polinomi di quarto grado:
		\item \(X^4-4x+2\) è irriducibile poiché è un \(2\)-eisenstain. La sua risolvente cubica è \(X^3-8X-16\) che è irriducibile e il suo discriminante non è un quadrato perfetto. Quindi \(G_g= S_3\) da cui \(G_f = S_4\).
		\item \(X^4+4X^2+2\) è irriducibile poiché è un \(2\)-eisenstain. La sua risolvente cubica è \((X-4)(X^2-8)\). Quindi \(G_g = C_2\) da cui \(G_f\) è \(D_4\) oppure \(C_4\). Osserviamo che \(\Q_g = \Q[\sqrt{2}]\), su cui \(f\) si scrive come \((X^2+2-\sqrt{2})(X^2+2+\sqrt{2})\). Quindi \(G_f=C_4\).
		\item \(X^4-2\) è irriducibile. La sua risolvente cubica è \(X\,(X^2+8)\). Quindi \(G_g=C_2\) e \(G_f\) è \(D_4\) oppure \(C_4\). Osserviamo che \(\Q_g = \Q[\sqrt{-2}]\), su cui \(f\) si può dimostrare essere ancora irriducibile. Quindi \(G_f = D_4\).
		\item \(X^4+10X^2+2\) è irriducibile. La sua risolvente cubica è \((X+10)(X+4)(X-4)\).
		      Quindi \(G_g = C_1\) da cui \(G_f = V\).
	\end{itemize}
\end{ese}
%%%%%%%%%%%%%%%%%%%%%%%%%
%POLINOMI DI GRADO PRIMO%
%%%%%%%%%%%%%%%%%%%%%%%%%
\section{Polinomi di grado primo}

In questo paragrafo ci occuperemo di polinomi irriducibili che hanno grado primo. In particolare studieremo il loro gruppo di Galois nel caso in cui abbiano precisamente \(p-2\) radici reali.

\begin{lem}
	Sia \(H\) un sottogruppo di \(S_p\). Supponiamo che \(H\) contenga una trasposizione e un \(p\)-ciclo.
	Allora \(H=S_p\).
\end{lem}

\begin{proof}
	Non fornita.
\end{proof}

\begin{prop}{}{}
	Sia \(f\in F[X]\) un polinomio irriducibile tale che \(\deg f = p\) e \(f\) ha \(p-2\) radici reali e \(2\) radici complesse. Allora
	\[
		\Gal(f) = S_p.
	\]
\end{prop}

\begin{proof}
	Sia \(G_f = \Gal(f)\).
	Vogliamo applicare il lemma precedente a \(G_f\).
	Supponiamo che \(\a\) sia una radice di \(f\), avremo
	\[
		F \subseteq F[\a] \subseteq F_f, \qquad\text{con }\big[F[\a]:F\big] = p.
	\]
	Quindi \(p \mid [F_f:F] = \#G_f\).
	Da un fatto di teoria dei gruppi\graffito{il fatto a cui facciamo riferimento è il teorema di Cauchy}, se \(p\mid \#G\) con \(G\) un gruppo finito, allora esiste \(g\in G\) tale che \(\ord(g)=p\).
	Quindi nel nostro caso esiste \(\s\in G_f\) tale che \(\ord(\s)=p\). Dal momento che \(G_f \le S_p\), \(\s\) è necessariamente un \(p\)-ciclo, infatti non esistono altri elementi di \(S_p\) con tale ordine.
	
	Per trovare la trasposizione, osserviamo che, per ipotesi, vi sono solo due radici complesse. In particolare se \(\a_1,\a_2\in\C\setminus\R\) sono tali radici, necessariamente \(\a_1=\conj{\a_2}\).
	Quindi se consideriamo l'automorfismo
	\[
		k\colon F_f \longrightarrow F_f, \a \longmapsto \conj{\a},
	\]
	avremo che
	\begin{align*}
		k(\a_1) & = \a_2; & k(\a_2) & = \a_1; & k(\a_j) & =\a_j,\,\fa j\ge 3.
	\end{align*}
	Quindi \(k=(1\ 2)\).
	Dal lemma segue che \(G_f = S_p\).
\end{proof}

\begin{oss}
	Per ogni \(p\) primo esiste sempre un polinomio con le proprietà descritte nella proposizione precedente.
	Se \(p=2,3\) è facile dare degli esempi, supponiamo quindi \(p\ge 5\).
	Siano \(n_1<\ldots<n_{p-2}\in\N\) pari e \(m>0\) pari. Definiamo
	\[
		g(X) = (X^2+m)(X-n_1)\cdot\ldots\cdot(X-n_{p-2}).
	\]
	Tale polinomio ha precisamente \(p-2\) radici reali. Cerchiamo di traslarlo opportunamente in modo da renderlo irriducibile senza cambiare il numero di radici reali.
	Definiamo
	\begin{align*}
		e & = \min\Set{\abs{g(x)} > 0 : g'(x) = 0}; & n & \in\N\text{ dispari tale che }\frac{2}{n}<e
	\end{align*}
	Prendiamo quindi
	\[
		f(X) = g(X)-\frac{2}{n}\in\Q[X]
	\]
	che p irriducibile.
	Infatti, per come abbiamo definito \(g\), si mostra facilmente che \(n\,f(X)=n\,g(X)-2\) è un \(2\)-eisenstain.
\end{oss}
%%%%%%%%%%%%%%%%%%%%%%%%%%%%%%%%%%%%
%PROBLEMA DI GALOIS INVERSO (CENNI)%
%%%%%%%%%%%%%%%%%%%%%%%%%%%%%%%%%%%%
\section{Problema di Galois inverso (cenni)}

Questo paragrafo vuole solo accennare in cosa consiste il problema di Galois inverso. Per una trattazione più approfondita si rimanda ad un testo più approfondito.

Il problema inverso consiste nel determinare, dato \(G\) un gruppo finito, se esiste \(f\in F[X]\) tale che
\[
	G_f \cong G.
\]
Tale questione resta un problema aperto nella sua forma più generale. In alcuni casi particolari è comunque possibile fornire una risposta certa.

\begin{teor}{Problema inverso per gruppi abeliani}{problemaInversoGruppiAbeliani}
	Sia \(G\) un gruppo abeliano.
	Allora esiste \(f\in F[X]\) tale che \(G_f \cong G\).
\end{teor}
%%%%%%%%%%%%%%%%%%%%%%%%%%%%%%%%%%%%%%%%%%
%
%LEZIONE 15/12/2016 - DECIMA SETTIMANA (2)
%
%%%%%%%%%%%%%%%%%%%%%%%%%%%%%%%%%%%%%%%%%%
%%%%%%%%%%%%%%
%CAMPI FINITI%
%%%%%%%%%%%%%%
\section{Campi finiti}

In questo paragrafo per denotare un generico campo finito useremo il simbolo \(\F\).
Cominciamo con il riepilogare alcune proprietà dei campi finiti.

\begin{pr}
	Esistono \(n\in\N\) e \(p\) primo tali che
	\[
		\# \F = p^n.
	\]
\end{pr}

\begin{pr}
	Se \(\F\) ha cardinalità \(p^n\), allora
	\[
		\F = \F_{p^n}.
	\]
\end{pr}

\begin{oss}
	In generale \(\F_{p^n}\neq \Z/p^n \Z\).
\end{oss}

\begin{pr}
	\(\F_{p^n}/\F_p\) è un'estensione finita di grado \(n\).
\end{pr}

\begin{pr}
	Per ogni \(x\in F_{p^n}\) si ha \(x^{p^n}=x\).
\end{pr}

\begin{pr}
	\(\F_{p^n}/\F_p\) è sempre un'estensione di Galois.
\end{pr}

\begin{proof}
	Il polinomio \(X^{p^n}-X\in \F_p[X]\) è separabile in quanto \((X^{p^n}-X)'=-1\). Quindi
	\[
		\F_{p^n} = (\F_p)_{X^{p^n}-X}
	\]
	è il campo di spezzamento di un polinomio separabile.
\end{proof}

\begin{pr}
	Due campi finiti \(\F_{q_1}\) e \(\F_{q_2}\) sono isomorfi se e solo se \(q_1=q_2\).
\end{pr}

\begin{pr}
	Per ogni \(q=p^n\) esiste un campo finito \(\F_q\) di ordine \(q\).
\end{pr}

\begin{proof}
	Consideriamo \(K=(\F_p)_{X^q-X}\) il campo di spezzamento di \(X^q-X\in \F_p[X]\).
	Definiamo \(S = \Set{\a \in K | \a^q = \a}\) l'insieme delle radici di \(X^q-X\). Osserviamo che \(\abs{S}=q\) in quanto sappiamo dalle proprietà precedenti che \(X^q-X\in \F_p[X]\) è un polinomio separabile.
	Certamente \(S\subseteq K\), se dimostriamo che \(S\) è un campo, esso deve necessariamente essere il campo di spezzamento di \(X^q-X\), da cui \(S=K\).
	Ora \(S\) è chiaramente chiuso rispetto alla moltiplicazione e al calcolo degli inversi. D'altronde è chiuso anche rispetto alla somma, infatti, presi \(\a,\b\in S\), per la formula sbagliata avremo
	\[
		(\a+\b)^q = (\a+\b)^{p^n} = \a^{p^n}+\b^{p^n} = \a+\b \implies \a+\b\in S.\qedhere
	\]
\end{proof}

\begin{teor}{Gruppo di Galois di \(\F_{p^n}/\F_p\)}{gruppoGaloisFpn}
	Sia \(q=p^n\) con \(p\) primo. Allora
	\[
		\Gal(\F_{p^n}/\F_p) \cong \Z/n\Z.
	\]
\end{teor}

\begin{proof}
	Sappiamo già che \(\#\Gal(\F_{p^n}/\F_p) = [\F_{p^n}:\F_p] = n\).
	Quindi, affinché \(\Gal(\F_{p^n}/\F_p)\cong \Z/n\Z\), ci basta dimostrare che \(\Gal(\F_{p^n}/\F_p)\) è ciclico. Dobbiamo quindi esibire un generatore.
	Consideriamo l'automorfismo di Frobenius:
	\[
		\Phi\colon \F_{p^n} \longrightarrow F_{p^n}, x \longmapsto x^p.
	\]
	Osserviamo che \(\Phi\) fissa gli elementi di \(\F_p\), quindi \(\Phi \in \Gal(\F_{p^n}/\F_p)\).
	A questo punto dobbiamo mostrare che \(\Phi\) ha ordine \(n\). Osserviamo che
	\[
		\Phi^k(X) := \underbrace{\Phi \circ \ldots \circ \Phi}_{k\text{ volte}}(X) = X^{p^k}.
	\]
	Quindi
	\[
		\Phi^n(x) = x^{p^n} = x,\,\fa x\in \F_{p^n}.
	\]
	Resta da mostrare che per ogni \(k<n,\Phi^k\neq id\), cioè che esite \(y\in \F_{p^n}\) tale che \(y^{p^k}\neq y\).
	Per un fatto di teoria dei gruppi, \(\F_{p^n}^*\) è ciclico.
	Sia \(y\) un generatore di \(\F_{p^n}^*\), allora
	\[
		\ord(y) = p^n-1, k<n \implies y^{p^k-1} \neq 1 \implies y^{p^k}\neq y.
	\]
	Quindi \(\ord(\Phi)=n\) e \(\langle \Phi \rangle = \Gal(\F_{p^n}/\F_p)\).
\end{proof}

\begin{oss}
	Alla luce delle proprietà precedenti, sappiamo che per ogni \(q=p^n\) con \(p\) primo, esiste \(\F_q\) il campo finito con \(q\) elementi. Inoltre due campi con \(q\) elementi sono isomorfi.
	Infine \(\F_q/\F_p\) è Galois e il teorema ci dice che
	\[
		\Gal(\F_q/\F_p) \cong \Z/n\Z.
	\]
\end{oss}

\begin{teor}{\(\F_{p^n}\) come estensione semplice}{FpnEstensioneSemplice}
	Consideriamo l'estensione \(\F_{p^n}/\F_p\). Allora esiste \(\z\in \F_{p^n}\) tale che
	\[
		\F_{p^n} = \F_p[\z].
	\]
\end{teor}

\begin{proof}
	Sia \(\z\) un generatore di \(\F_{p^n}^*\), che sappiamo esistere per un fatto di teoria dei gruppi. Segue immediatamente che \(\F_p[\z] = \F_{p^n}\), infatti
	\[
		\F_{p^n} = \Set{0,\z,\z^2,\ldots,\z^{p^n-1}}.\qedhere
	\]
\end{proof}

\begin{oss}
	Più in generale, tramite il teorema dell'elemento primitivo, si può dimostrare che se \(K/\Q\) è finita, allora esiste \(\a\in K\) tale che \(K=\Q[\a]\).
\end{oss}

\begin{teor}{Sottocampi di \(\F_{p^n}\)}{sottocampiFpn}
	Consideriamo l'estensione \(\F_{p^n}/\F_p\). Per ogni \(k\mid n\) esiste un unico sottocampo \(\F_{p^k}\) con \(p^k\) elementi.
\end{teor}

\begin{proof}
	Abbiamo mostrato che \(\Gal(\F_{p^n}/\F_p)=\Z/n\Z = \langle \Phi \rangle\). Per le proprietà dei gruppi ciclici, sappiamo che per ogni divisore \(k\) dell'ordine di \(\langle \Phi \rangle\) vi è un solo sottogruppo di indice \(k\).
	Quindi per ogni \(k\mid n\) avremo il sottogruppo \(\langle \Phi^k \rangle \cong \Z/\frac{n}{k}\Z\) a cui corrisponde il sottocampo
	\[
		\F_{p^n}^{\langle \Phi^k \rangle} = \F_{p^k}.\qedhere
	\]
\end{proof}

\begin{oss}
	Vale anche il viceversa, cioè se \(\F_{p^k}\subseteq \F_{p^n}\) allora \(k\mid n\).
\end{oss}

\begin{defn}{Funzione enumeratrice dei polinomi irriducibili in \(\F_p\)}{funzioneNumeroPolinomiIrriducibiliFp}
	Sia \(p\) primo. Definiamo la \emph{funzione che enumera i polinomi irriducibili di grado \(d\)} in \(\F_p\) come
	\[
		N_d(p) = \#\Set{f\in\irr\big(\F_p[X]\big) | \deg f = d}.
	\]
\end{defn}

\begin{prop}{Numero di polinomi irriducibili in \(\F_p\)}{numeroPolinomiIrriducibiliFp}
	Sia \(p\) primo. Allora
	\[
		\sum_{d\mid n} d\,N_d(p) = p^n.
	\]
\end{prop}

\begin{proof}
	Consideriamo \(f(X) = X^{p^n}-X\in \F_p[X]\). Se mostriamo
	\begin{equation*}
		f(X) = \prod_{\substack{f\in\irr(\F_p[X])\\\deg f \mid n}} f, \tag{\(\star\)}
	\end{equation*}
	seguirebbe
	\[
		p^n = \deg f = \deg \prod_{d\mid n}\prod_{\substack{f\in\irr(\F_p[X])\\\deg f = d}} f = \sum_{d\mid n}N_d(p)\,d.
	\]
	Mostriamo quindi \((\star)\).
	Sia \(g\) un fattore irriducibile di \(X^{p^n}-X\) e sia \(\a\) una radice di \(g\). Avremo
	\[
		\F_p \subseteq \F_p[\a] \subseteq \F_{p^n} \implies \deg g = \big[\F_p[\a]:\F_p\big] \mid n.
	\]
	In particolare, dal momento che \(X^{p^n}-X\) è il prodotto di tali fattori irriducibili ed è anche separabile, segue
	\[
		X^{p^n}-X \mid \prod_{\substack{f\in\irr(\F_p[X])\\\deg f \mid n}} f.
	\]
	Per concludere basta dimostrare che se \(h\in \irr\big(\F_p[X]\big)\) e \(\deg h\mid n\), allora
	\[
		h \mid X^{p^n}-X.
	\]
	Sia \(\b\) una radice di \(h\). Per la corrispondenza e l'unicità dei campi finiti, \(\F_p[\b]\) si inietta isomorficamente in \(\F_{p^n}\). Nel sottocampo di \(\F_{p^n}\) isomorfo a \(\F_p[\b]\) ci sono tutte le radici di \(h\), le quali sono in particolare radici di \(X^{p^n}-X\).
\end{proof}

\begin{oss}
	Se \(n=l\) primo, allora la formula si riduce a 
	\[
		N_1(p) + l\,N_l(p) = p^l \implies N_l(p) = \frac{p^l-p}{l},
	\]
	poiché chiaramente \(N_1(p)=p\).
\end{oss}