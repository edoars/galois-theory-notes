%!TEX root = ../main.tex
\chapter{Definizioni e risultati di base}
%%%%%%%%%%%%%%%%%%%%%%%%%%%%%%%%%%%%%%%%%
%
%LEZIONE 27/09/2016 - PRIMA SETTIMANA (2)
%
%%%%%%%%%%%%%%%%%%%%%%%%%%%%%%%%%%%%%%%%%
%%%%%%%%
%ANELLI%
%%%%%%%%
\section{Anelli}

\begin{defn}{Anello}{anello}\index{Anello}
	Un \emph{anello} è un insieme \(A\) dotato di due operazioni \(+\) e \(\cdot\) tali che
	\begin{enumerate}
		\item \((A,+)\) è un gruppo commutativo;
		\item \((A,\cdot)\) è un monoide commutativo;
		\item la moltiplicazione è distributiva rispetto alla somma.
	\end{enumerate}
\end{defn}

\begin{notz}
	In questo corso un anello, a meno di esplicitarlo altrimenti, si intende sempre unitario e commutativo.
\end{notz}

\begin{defn}{Sottoanello}{sottoanello}\index{Sottoanello}
	Sia \(A\) un anello e \(B\subseteq A\).
	\(B\) si dice \emph{sottoanello} di \(A\) se
	\begin{itemize}
		\item \((B,+)\) è un sottogruppo di \((A,+)\);
		\item \(B\) è chiuso rispetto al prodotto;
		\item \(1_A\) appartiene a \(B\).
	\end{itemize}
\end{defn}

\begin{oss}
	In particolare un sottoanello costituisce un anello.
	D'altronde il viceversa è falso, infatti \(A=\Z\times \Z\) è un anello rispetto alle operazioni canoniche la cui identità è \((1,1)\).
	Se considero \(S = \Set{(x,0)| x\in \Z}\) avrò che \(S\) è un anello e \(S\subseteq A\), ma \(S\) non è un sottoanello di \(A\) in quanto \((1,1)\not\in S\), la cui identità è invece \((1,0)\).
\end{oss}

\begin{defn}{Omomorfismo di anelli}{omomorfismoAnelli}\index{Omomorfismo!di anelli}
	Siano \(A,A'\) anelli.
	Un \emph{omomorfismo di anelli} \(\j\colon A \to A'\) è una mappa che mantiene le operazioni, ovvero tale che per ogni \(a,b\in A\)
	\[
		\j(a+b)=\j(a)+\j(b), \qquad \j(a\,b)=\j(a)\j(b), \qquad \j(1_A) = 1_{A'}.
	\]
\end{defn}

\begin{ese}
	Riprendendo l'esempio dell'osservazione precedente avremo che
	\[
		\j\colon S \hookrightarrow A, (x,0) \mapsto (x,0),
	\]
	non è un omomorfismo in quanto
	\[
		\j(1_S) = \j((1,0)) = (1,0) \neq 1_A.
	\]
\end{ese}

\begin{defn}{Dominio di integrità}{dominio}\index{Dominio di integrità}
	Un anello \(A\) si definisce \emph{dominio di integrità} se il prodotto di elementi non nulli è sempre non nullo, ovvero
	\[
		a\,b = 0 \implies a=0\text{ oppure }b=0,\,\fa a,b\in A.
	\]
\end{defn}

\begin{notz}
	Spesso la sola parola dominio viene utilizzata per i domini di integrità.
\end{notz}

\begin{ese}
	\(\Z \times \Z\) non è un dominio di integrità, infatti
	\[
		(1,0)(0,1) = (0,0) \qquad\text{con}\qquad (1,0),(0,1) \neq (0,0).
	\]
\end{ese}

\begin{defn}{Ideale}{ideale}\index{Ideale}
	Sia \(A\) un anello e sia \(I\subseteq A\).
	\(I\) si dice \emph{ideale} di \(A\) se
	\begin{itemize}
		\item \((I,+)\) è un sottogruppo di \((A,+)\);
		\item \(I\) è chiuso rispetto alla moltiplicazioni di elementi in \(A\), ovvero
		      \[
			      a\,x \in I,\,\fa x \in I,\,\fa a \in A.
		      \]
	\end{itemize}
\end{defn}

\begin{pr*}
	Se \(A\) è un anello e \(I\subseteq A\) è un ideale,
	\[
		\frac{A}{I} = \Set{a+I : a\in A} \qquad\text{con }a+I = \Set{a+x : x\in I}\subseteq A,
	\]
	è una partizione di \(A\) e costituisce un anello con le operazioni indotte sulle classi laterali.
\end{pr*}

\begin{defn}{Ideale generato}{idealeGenerato}\index{Ideale!generato}
	Sia \(A\) un anello e siano \(\a_1,\ldots,\a_r\in A\).
	Definiamo
	\[
		(\a_1,\ldots,\a_r) = \Set{x_1\a_1 + \ldots + x_r\a_r | x_1, \ldots, x_r \in A},
	\]
	come l'\emph{ideale generato} da \(\a_1,\ldots,\a_r\).
\end{defn}

\begin{defn}{Anello a ideali principali}{PID}\index{Anello!a ideali principali}
	Un anello \(A\) si dice \emph{a ideali principali} se per ogni ideale \(I\subseteq A\), si ha che \(I\) è generato da un elemento di \(A\), ovvero
	\[
		\ex a\in A: I = (a).
	\]
\end{defn}
%%%%%%%
%CAMPI%
%%%%%%%
\section{Campi}

\begin{defn}{Campo}{campo}\index{Campo}
	Un anello \(K\) si definisce \emph{campo} se ogni suo elemento non nullo è invertibile, ovvero
	\[
		\fa x \in K, x\neq 0 \,\ex y\in K: x\,y = 1_K.
	\]
\end{defn}

\begin{oss}
	In altre parole un campo è un insieme \(K\) costituito da due operazioni \(+\) e \(\cdot\) tali che
	\begin{enumerate}
		\item \((K,+)\) è un gruppo abeliano;
		\item \((K\setminus\{0\},\cdot)\) è un gruppo abeliano;
		\item la moltiplicazione è distributiva rispetto alla somma.
	\end{enumerate}
\end{oss}

\begin{ese}
	\(\Q,\R\) e \(\C\) sono alcuni esempi di campi.
\end{ese}

\begin{oss}
	Vi sono moltissimi campi che possono essere costruiti fra \(\Q\) ed \(\R\) od oltre \(\C\).
	D'altronde non ve ne è nessuno fra \(\R\) e \(\C\).
\end{oss}

\begin{ese}
	L'estensione algebrica \(\Q(\sqrt{2})\) di \(\Q\) definita come
	\[
		\Q(\sqrt{2}) = \Set{a+b\,\sqrt{2} | a,b\in \Q},
	\]
	è un campo.
	Si mostra facilmente che ogni suo elemento ha inverso, infatti
	\[
		\frac{1}{a+\sqrt{2}b} = \frac{a}{a^2-2b^2} - \frac{b}{a^2-2b^2}\sqrt{2} \in \Q(\sqrt{2}).
	\]
\end{ese}

\begin{ese}
	L'esensione trascendente \(\Q(\p)\) di \(\Q\) definita come
	\[
		\Q(\p) = \Set{ \frac{a_0+a_1\p + \ldots + a_n \p^n}{b_0 + b_1\p + \ldots + b_m \p^m} | a_0,\ldots,a_n,b_0,\ldots,b_m \in \Q},
	\]
	è un campo.
\end{ese}

\begin{defn}{Sottocampo}{sottocampo}\index{Sottocampo}
	Sia \(K\) un campo e sia \(L\subseteq K\).
	\(L\) si dice \emph{sottocampo} di \(K\) se è un suo sottoanello ed inoltre costitutisce un campo.
\end{defn}

\begin{oss}
	Se \(L\) è un sottocampo di \(K\) allora \(K\) è anche uno spazio vettoriale su \(L\).
	Infatti se \(\a\in L,k\in K\), allora \(\a \cdot k = \a\,k\).
	su uno spazio vettoriale posso anche parlare di dimensione \(\dim_L K\).
\end{oss}

\begin{notz}
	Un'inclusione di campi \(L\subseteq K\) si chiama \emph{estensione di campi} e se ne definisce il \emph{grado} come
	\[
		[K:L] := \dim_L K.
	\]
\end{notz}

\begin{ese}
	Rifacendoci a due esempi precedenti abbiamo
	\[
		[\Q(\sqrt{2}):\Q] = 2 \qquad\text{e}\qquad [\Q(\p):\Q] = +\infty.
	\]
\end{ese}

\begin{prop}{Caratterizzazione dei campi tramite ideali}{caratterizzazioneCampiIdeali}
	Sia \(A\) un anello.
	Allora \(A\) è un campo se e soltanto se per ogni \(I\subseteq A\) ideale risulta
	\[
		I = (0) \qquad\text{oppure}\qquad I=A.
	\]
\end{prop}

\begin{proof}
	\graffito{\(\Leftarrow)\)}Sia \(a\in A\setminus\{0\}\).
	Dobbiamo esibire un inverso di \(a\).
	Consideriamo l'ideale da esso generato \((a)\), chiaramente
	\[
		(a) \neq 0 \implies (a)=A \implies 1\in (a).
	\]
	Ovvero esiste \(b\in A\) tale che \(a\,b=1\), quindi \(a\) è invertibile.

	\graffito{\(\Rightarrow)\)}Sia \(I\subseteq A\) un ideale tale che \(I\neq (0)\).
	Allora per ogni \(a\in A\) possiamo fissare \(x\in I, x\neq 0\) e scrivere
	\[
		a = a\,x^{-1}x.
	\]
	D'altronde \(a\,x^{-1}\in A\) e \(x\in I\), da cui
	\[
		a\,x^{-1}x \in I \implies a \in I,\,\fa a \in A.
	\]
	Ovvero \(A\subseteq I\) che implica immediatamente \(A=I\).
\end{proof}
%%%%%%%%%%%%%%%%%%%%%%%%%%%%%%%%%%%%%%%%%
%
%LEZIONE 28/09/2016 - PRIMA SETTIMANA (3)
%
%%%%%%%%%%%%%%%%%%%%%%%%%%%%%%%%%%%%%%%%%
\begin{prop}{Omomorfismi di campi}{omomorfismiCampi}
	Sia \(\j\colon F_1 \to F_2\) un omomorfismo di campi.
	Allora \(\j\) è iniettivo.
\end{prop}

\begin{proof}
	Sappiamo che \(\Ker \j \subseteq F_1\) è un ideale.
	D'altronde \(F_1\) è un campo, quindi, per la \autoref{pr:caratterizzazioneCampiIdeali},
	\[
		\Ker \j = (0) \qquad\text{oppure}\qquad \Ker \j = F_1.
	\]
	Ma \(1_{F_1}\not\in \Ker\j\) in quanto \(\j(1_{F_1})=1_{F_2}\neq 0_{F_2}\).
	Quindi \(\Ker \j = (0)\), ovvero \(\j\) è iniettivo.
\end{proof}
%%%%%%%%%%%%%%%%%%%%%%%%%%%%
%CARATTERISTICA DI UN CAMPO%
%%%%%%%%%%%%%%%%%%%%%%%%%%%%
\section{Caratteristica di un campo}

Sia \(F\) un campo, si mostra facilmente che la mappa
\[
	\j\colon \Z \to F, n\mapsto \overbrace{1_F+1_F+\ldots+1_F}^{n \text{volte}} =: n\,1_F, -1 \mapsto -1_F,
\]
è un omomorfismo di anelli. Pertanto il suo nucleo \(\Ker \j\) è un ideale di \(\Z\).

\begin{defn}{Caratteristica}{caratteristica}\index{Caratteristica}
	Sia \(A\) un anello, si definisce \emph{caratteristica} di \(A\) il più piccolo naturale \(n\in \N\) tale che
	\[
		\underbrace{1_A+1_A+\ldots+1_A}_{n \text{volte}} = 0_A.
	\]
\end{defn}

\begin{notz}
	Se tale naturale non esiste si dice che \(A\) ha caratteristica \(0\) per definizione.
\end{notz}

\begin{teor}{Caratteristica di un campo}{caratteristicaCampo}\index{Caratteristica!di un campo}
	Sia \(F\) un campo.
	Allora la caratteristica di \(F\) è zero oppure un numero primo.
\end{teor}

\begin{proof}
	Consideriamo nuovamente l'omomorfismo \(\j\) introdotto all'inizio del paragrafo e analizziamo due casi distinti:
	\begin{itemize}
		\item Se \(\Ker \j=(0)\), allora
		      \[
			      n\,1_F = 0 \implies n = 0.
		      \]
		      Per cui gli elementi non nulli di \(\Z\) vengono mappati in elementi invertibili di \(F\), ne segue che \(\j\) può essere esteso a \(\Q\) tramite
		      \[
			      \Q \to F, \frac{n}{m} \mapsto (n\,1_F)(m\,1_F)^{-1},
		      \]
		      ovvero in questo caso \(F\) contiene una copia isomorfa a \(\Q\) ed ha caratteristica zero.
		\item Se \(\Ker \j \neq (0)\) allora esiste \(m\neq 0\) tale che \(m\,1_F = 0_F\) e si avrebbe
		      \[
			      p = \Char F = \min\Set{m | m\,1_F = 1_F+1_F + \ldots +1_F =0},
		      \]
		      è primo.
		      Infatti se per assurdo \(p=a\,b\) si avrebbe \(p\,1_F = a\,b\,1_F = (a\,1_F)(b\,1_F)\).
		      D'altronde \(F\) è in particolare un dominio, per cui
		      \[
			      p\,1_F = 0 \implies a\,1_F = 0 \qquad\text{oppure}\qquad b\,1_F = 0,
		      \]
		      ma ciò è assurdo per la minimalità di \(p\).

		      Inoltre in questo caso si ha \(\Ker \j = (p)\subset\Z\), per cui il Teorema Fondamentale degli Omomorfismi definisce un'inclusione
		      \[
			      \frac{\Z}{(p)} = \frac{\Z}{p\Z} \hookrightarrow F, n \pmod{p} \mapsto n\,1_F.
		      \]
		      Ovvero \(F\) contiene una copia isomorfa a \(\F_p\) e ha caratteristica \(p\).
	\end{itemize}
\end{proof}

\begin{notz}
	Quando \(F\) ha caratteristica \(p\) diciamo che \(\F_p\) è il \emph{sottocampo fondamentale} di \(F\).
\end{notz}

\begin{prop}{Binomio di Newton nei campi}{binomioNewtonCampi}
	Se \(F\) è un campo di caratteristica \(p\) allora
	\[
		(a+b)^p = a^p+b^p,\,\fa a,b\in F.
	\]
\end{prop}

\begin{proof}
	Il binomio di Newton
	\[
		(a+b)^n = \sum_{k=0}^{n} \binom{n}{k} a^{n-k}b^k
	\]
	è valido in ogni anello commutativo. Ora se \(n=p\) si ha \(p\mid \binom{p}{k}\) per ogni \(k=1,\ldots,p-1\).
	Quindi se \(F\) ha caratteristica \(p\) avremo
	\[
		p \mid \binom{p}{k} \implies \binom{p}{k} 1_F = m\,p\,1_F = m\,(p\,1_F) = 0_F,
	\]
	da cui, sostituendo nell'espressione del binomio di Newton, si giunge alla tesi.
\end{proof}

\begin{oss}
	In generale vale
	\[
		(a+b)^{p^n} = a^{p^n}+b^{p^n},\,\fa n\ge 1.
	\]
	Per cui la mappa \(F\to F, x \mapsto x^p\) risulta essere un omomorfismo, detto \emph{Endomorfismo di Frobenius}.
	Tale endomorfismo risulta essere un automorfismo quando \(F\) è finito.
\end{oss}
%%%%%%%%%%%%%%%%%%%%
%ANELLI DI POLINOMI%
%%%%%%%%%%%%%%%%%%%%
\section{Anelli di polinomi}

Se \(F\) è un campo possiamo definire il seguente insieme
\[
	F[X] = \Set{\sum_{j=0}^k a_j X^j | a_j\in F}.
\]
Inoltre se consideriamo due elementi
\[
	f(X) = \sum_{j=0}^n a_j X^j \qquad\text{e}\qquad g(X) = \sum_{j=0}^m b_j X^j,
\]
possiamo definire le operazioni di somma
\[
	(f+g) = \sum_{j=0}^{\max\{n,m\}}(a_j+b_j)X^j \qquad\text{con }\begin{cases}a_j=0 & \text{se }j>n\\b_j=0 & \text{se }j>m\end{cases}
\]
e di prodotto
\[
	(f\,g) = \sum_{j=0}^{m+n} c_j X^j \qquad\text{con }c_j = \sum_{\substack{h+k=j\\0\le h\le n\\0\le k\le m}} a_h + b_k.
\]

\begin{defn}{Anello di polinomi}{anelloPolinomi}\index{Anello di polinomi}
	Sia \(F\) un campo. L'insieme \(F[X]\) si definisce \emph{anello di polinomi} nell'indeterminata \(X\) a coefficienti in \(F\).
\end{defn}

\begin{oss}
	Si osservi che \(F[X]\) è un anello rispetto alle operazioni definite sopra, inoltre risulta \(F\subseteq F[X]\) e \(F[X]\) un dominio.
\end{oss}

\begin{pr}[Divisione di polinomi]
	Siano \(f,g\in F[X]\) con \(g\neq 0\).
	Allora esistono unici \(q(X),r(X)\in F[X]\) tali che
	\[
		f(X) = q(X)g(X) + r(X) \qquad\text{con }r(X)=0 \text{ oppure }\deg r< \deg g.
	\]
\end{pr}

\begin{oss}
	\(F[X]\) è pertanto un dominio euclideo con il grado dei polinomi come norma.
	In particolare è anche un dominio a fattorizzazione unica ed esiste sempre il MCD di due elementi.
\end{oss}

\begin{pr}
	Siano \(f(X)\in F[X]\) e \(a\in F\). Allora esiste unico \(q(X)\in F[X]\) tale che
	\[
		f(X) = (X-a)q(X) + c \qquad\text{con }c=f(a).
	\]
\end{pr}

\begin{oss}
	Se \(a\) è una radice di \(f\), ovvero \(f(a)=0\), allora
	\[
		(X-a) \mid f(X),
	\]
	da ciò segue inoltre che \(f\) ha al più \(\deg f\) radici.
\end{oss}

\begin{pr}[Algoritmo euclideo]
	Siano \(f,g\in F[X]\) e supponiamo che \(d(X)=\big(f(X),g(X)\big)\).
	Tramite l'algoritmo euclideo delle divisioni è possibile costruire \(a(X),b(X)\in F[X]\) tali che
	\[
		a(X)f(X) + b(X)g(X) = d(X) \qquad\text{con }\deg a < \deg g \text{ e } \deg b < \deg f.
	\]
\end{pr}

\begin{defn}{Campo dei quozienti dei polinomi}{campoQuozientiPolinomi}\index{Campo dei quozienti!dei polinomi}
	Dal momento che \(F[X]\) è un dominio di integrità possiamo considerare il suo \emph{campo dei quozienti} \(F(X)\).
	Esso è costituito dai quozienti \(f/g\), dove \(f,g\in F[X]\) e \(g\neq 0\).
\end{defn}
%%%%%%%%%%%%%%%%%%%%%%%%%%%%%
%FATTORIZZAZIONE DI POLINOMI%
%%%%%%%%%%%%%%%%%%%%%%%%%%%%%
\section{Fattorizzazione di Polinomi}

In questo paragrafo studieremo in quali casi è possibile determinare la riducibilità dei polinomi.

In questo corso quando diciamo che \(f\in \Z[X]\) è irriducibile si intende che per ogni \(g\mid f\) si ha \(\deg g=0\) oppure \(\deg g = f\).

\begin{prop}{Radici razionali di un polinomio a coefficienti interi}{radiciPolinomioIntero}
	Sia \(f(X) = a_0+a_1 X + \ldots + a_m X^m \in \Z[X]\) e supponiamo che \(q=N/D, N,D\in \Z, (N,D)=1\) sia una radice di \(f\).
	Allora
	\[
		N \mid a_0 \qquad\text{e}\qquad D\mid a_m.
	\]
\end{prop}

\begin{proof}
	Per ipotesi \(f(q)=0\). Se all'espressione di \(f(q)\) semplifichiamo il denominatore, otteniamo
	\[
		a_0 D^m + a_1 D^{m-1}N + \ldots + a_{m-1}D\,N^{m-1}+a_m N^m = 0 \implies D\,(a_0 D^{m-1}+\ldots+a_{m-1}) = -a_m N^m,
	\]
	da cui \(D\mid a_m N^m\). D'altronde \((N,D)=1 \implies D\mid a_m\).

	Analogamente si mostra che \(N\mid a_0\).
\end{proof}

\begin{ese}
	Consideriamo il polinomio \(f(X)=X^3+a\,X+1,a\in \Z\).
	Per la proposizione le uniche possibili radici razionali di \(f\) sono \(x=\pm 1\), dove
	\[
		f(1) = a+2 \qquad\text{e}\qquad f(-1) = -a.
	\]
	Per cui se \(a\neq 0,-2\) allora \(f\) è irriducibile.
\end{ese}

\begin{prop}{Lemma di Gauss}{lemmaGauss}
	Sia \(f(X) \in \Z[X]\) e supponiamo che \(f\) si fattorizzi in modo non banale in \(\Q[X]\).
	Allora \(f\) si fattorizza in modo non banale anche in \(\Z[X]\).
\end{prop}

\begin{proof}
	Per ipotesi \(f= h\,g\) con \(h,g\in \Q[X]\) divisori propri di \(f\). Certamente esisteranno \(m,n\in \Z\) tali che
	\[
		m\,h(X) = h_1(X) \in \Z[X] \qquad\text{e}\qquad n\,g(X) = g_1(X) \in \Z[X],
	\]
	da cui
	\begin{equation}\label{eq:lemmaGauss}
		m\,n\,f(X) = h_1(X)g_1(X). \tag{\(\star\)}
	\end{equation}
	Vogliamo mostrare di poter assumere che \(m\,n = 1\).

	Se \(p\mid n\,m\) leggiamo \(\eqref{eq:lemmaGauss}\) in \(\F_p[X]\), così da ottenere \(\bar{h}_1\bar{g}_1 \equiv_p 0\).
	D'altronde \(\F_p[X]\) è un dominio, per cui \(\bar{h}_1(X) \equiv_p 0\) oppure \(\bar{g}_1(X) \equiv_p 0\).
	Assumiamo \(\bar{h}_1 \equiv_p 0\), ciò significa che tutti i coefficienti di \(h_1\) sono divisibili per \(p\). Quindi
	\[
		m\,h(X) = h_1(X) = p\,h_2(X) \implies \frac{m\,n}{p}\,f(X) = h_2(X)g_1(X),
	\]
	iterando il procedimento si giunge alla tesi.
\end{proof}
%%%%%%%%%%%%%%%%%%%%%%%%%%%%%%%%%%%%%%%%%%%
%
%LEZIONE 03/10/2016 - SECONDA SETTIMANA (1)
%
%%%%%%%%%%%%%%%%%%%%%%%%%%%%%%%%%%%%%%%%%%%
\begin{prop}{Fattori monici di un polinomio monico a fattori interi}{fattoriMoniciPolinomioMonico}
	Sia \(f(X)\in\Z[X]\) un polinomio monico. Supponiamo che \(g\mid f\) con \(g\in \Q[X]\) monico.
	Allora \(g(X)\in\Z[X]\).
\end{prop}

\begin{proof}
	Scriviamo \(f=g\,h\) con \(g,h\in\Q[X]\) monici.
	Sappiamo, tramite lo stesso argomento della \hyperref{pr:lemmaGauss}{proposizione precedente}, che esistono \(m,n\in \Z\) tali che \(m\,g,n\,h\in\Z[X]\), consideriamo inoltre \(m,n\) tali che abbiano un numeri di fattori primi minimi.
	Vogliamo ottenere una contraddizione mostrando che se \(p\mid m\,n\) allora \(m\), oppure \(n\), non sarebbero minimali rispetto alla proprietà di avere un numero minimo di fattori.

	Supponiamo quindi che \(p\mid m\,n\) con \(p\) primo, allora
	\[
		m\,g\cdot n\,h = m\,n\,f \implies m\,g \cdot n\,h \equiv_p 0.
	\]
	Siccome \(\F_p[X]\) è un dominio, otteniamo \(m\,g\equiv_p 0\) oppure \(n\,h \equiv_p 0\). Assumiamo che \(m\,g\equiv_p= 0\), in tal caso si avrebbe \(p\mid m\) in quanto \(g\) è monico per ipotesi, da cui
	\[
		\frac{m}{p}g(X) \in \Z[X],
	\]
	che è assurdo per la minimalità di \(m\).
\end{proof}

\begin{prop}{Criterio di Eisenstein}{criterioEisenstein}
	Sia \(f(X) = a_m X^m+a_{m-1}X^{m-a} + \ldots + a_1 X + a_0 \in \Z[X]\) e supponiamo che esista \(p\) primo tale che
	\begin{enumerate}
		\item \(p\) non divide \(a_m\).
		\item \(p\) divide \(a_j\) per ogni \(j\in\{0,\ldots,m-1\}\).
		\item \(p^2\) non divide \(a_0\).
	\end{enumerate}
	Allora \(f\) è irriducibile in \(\Q[X]\).
\end{prop}

\begin{proof}
	Se per assurdo fosse
	\[
		a_m X^m + \ldots + a_1 X + a_0 = (b_r X^r + \ldots + b_1 X+b_0)(c_s X^s + \ldots + c_1 X + c_0).
	\]

	Dal momento che \(p\), ma non \(p^2\), divide \(a_0=b_0 c_0\), si avrebbe che \(p\) deve dividere necessariamente \(b_0\) oppure \(c_0\), assumiamo \(b_0\). Inoltre da
	\[
		a_1 = b_0 c_1+ b_1 c_0,
	\]
	deduciamo che \(p\mid b_1\). Analogamente da
	\[
		a_2 = b_0 c_2 + b_1 c_1 + b_2 c_0,
	\]
	deduciamo che \(p\mid b_2\).
	Iterando tale procedimento otteniamo che \(p\) divide \(b_0,b_1,\ldots,b_r\) che è assurdo per l'ipotesi che \(p\nmid a_m\).
\end{proof}

\begin{oss}
	Le proposizioni che abbiamo dimostrato finora in questo paragrafo sono ancora valide se al posto di \(\Z\) consideriamo un qualsiasi altro dominio a fattorizzazione unica.
\end{oss}

\begin{pr}
	Sia \(f(X)\in \Z[X]\) e siano \(a,b\in \Q,a\neq 0\). Allora \(f(X)\) è irriducibile se e solo se \(f(a\,X+b)\) è irriducibile.
\end{pr}

\begin{proof}
	Supponiamo che \(f(a\,X+b)\) sia irriducibile, se per assurdo fosse \(f(X)=g(X)h(X)\) si avrebbe
	\[
		f(a\,X+b) = g(a\,X+b)h(a\,X+b),
	\]
	che è chiaramente assurdo.
	Analogamente si mostra il viceversa, infatti
	\[
		F(X):=f(a\,X+b) \implies f(X) = F \Big(\frac{1}{a}\,X- \frac{b}{a}\Big),
	\]
	d'altronde abbiamo già mostrate che \(F(X)\) irriducibile implica \(F(1/a\,X-b/a)\) irriducibile.
\end{proof}

\begin{ese}
	Consideriamo il \emph{\(p\)-esimo polinomio ciclotomico}
	\[
		\phi_p(X) = \frac{X^p-1}{X-1} = 1+X+\ldots+X^{p-1} = \prod_{j=1}^{p-1}\Big(X-e^{\frac{2\p\,i\,j}{p}}\Big) \in \Z[X]
	\]
	Per mostrare che \(\phi_p(X)\) è irriducibile vogliamo sfruttare il criterio di Eisenstein. Scriviamo \(\phi_p(X+1)\):
	\[
		\phi_p(X+1) = \frac{(X+1)^p-1}{X} = X^{p-1}+p\,X^{p-2} + \binom{p}{2}X^{p-3} + \ldots + \binom{p}{p-2}X + \binom{p}{p-1}\graffito{tramite il binomio di Newton},
	\]
	otteniamo quindi che \(\phi_p(X+1)\) è un \(p\)-eisenstein, per cui \(\phi_p(X+1)\) è irriducibile e di conseguenza \(\phi_p(X)\) è irriducibile.
\end{ese}

\begin{teor}{Irriducibilità in \(\Z[X]\) è deterministico}{irriducibilitàDeterministica}
	Sia \(f(X)\in \Z[X]\), allora esiste un algoritmo per fattorizzare \(f\).
	Ovvero l'irriducibilità di un polinomio in \(\Z[X]\) è un problema deterministico.
\end{teor}

\begin{proof}
	Possiamo assumere che \(f\) sia monico a meno di moltiplicare per una costante, per cui
	\[
		f(X) = X^m + a_{1}X^{m-1} + \ldots + a_m, \qquad\text{con }a_i \in \Z.
	\]
	Dal Teorema Fondamentale dell'Algebra sappiamo che esistono \(\a_1,\ldots,\a_m\in\C\) tali che
	\[
		f(X) = \prod_{j=1}^m (X-\a_j).
	\]
	Osserviamo che dall'identità
	\[
		0 = f(\a_j) = \a_j^m + a_1 \a_j^{m-1} + \ldots + a_m,
	\]
	si deduce che \(\abs{\a_j}\) è limitata e può essere stimata in termini dei soli coefficienti di \(f\). Infatti avremo
	\[
		\abs{\a_j} \le \bigg\lvert \frac{a_m}{\a_j^{m-1}} \bigg\rvert + \bigg\lvert \frac{a_{m-1}}{\a_j^{m-2}} \bigg\rvert + \ldots + \abs{a_1} \implies \abs{\a_j} \le \sum_{k=1}^m \frac{\abs{a_k}}{\abs{\a_j}^{k-1}}
	\]
	Da cui
	\[
		\abs{\a_j} \ge 1 \implies \abs{\a_j} \le \sum_{k=1}^m \abs{a_k},
	\]
	ovvero
	\[
		\abs{\a_j} \le \max\Set{1,\sum_{k=1}^m \abs{a_k}}, \,\fa j=1,\ldots,m.
	\]
	Ora se \(g(X)\) è un fattore monico di \(f(X)\), allora le sue radici saranno un sottoinsieme di quelle di \(f\) e i suoi coefficienti saranno polinomi simmetrici nelle sue radici. Per cui i moduli dei coefficienti di \(g\) saranno limitati in termini dei coefficienti di \(f\).
	Dal momento che essi sono anche interi, ne deduciamo che esistono solo un numero finito di possibilità per \(g(X)\).
	Per cui, per trovare i fattori di \(f(X)\), dobbiamo analizzare un numero finito di casi.
\end{proof}

\begin{oss}
	Tale procedimento può essere esteso anche ai polinomi in \(\Q[X]\). Infatti se \(f\in \Q[X]\) possiamo renderlo monico tramite la moltiplicazione per un razionale e infine sostituirlo con
	\[
		F_D(f):= D^{\deg f} f \left( \frac{X}{D} \right),
	\]
	dove \(D\) è il mcm dei denominatore dei coefficienti di \(f\). Abbiamo così ottenuto un polinomio monico a coefficienti interi che ha le stesse radici di quello di partenza.
\end{oss}

\begin{ese}
	Se \(f(X) = X-1/2\) possiamo scrivere
	\[
		F_2(f) = 2^1 \left( \frac{X}{2}-\frac{1}{2} \right) = X-1.
	\]
\end{ese}
%%%%%%%%%%%%%%%%%%%%%%%%%%%%%%%%%%%%%%%%%%%
%
%LEZIONE 04/10/2016 - SECONDA SETTIMANA (2)
%
%%%%%%%%%%%%%%%%%%%%%%%%%%%%%%%%%%%%%%%%%%%
%%%%%%%%%%%%%%%%%%%%%
%ESTENSIONE DI CAMPI%
%%%%%%%%%%%%%%%%%%%%%
\section{Estensione di campi}

\begin{defn}{Estensione di campi}{estensioneCampi}\index{Estensione di campi}
	Se \(E,F\) sono campi e \(F\subseteq E\) è un sottocampo, diciamo che \(E\) è un'\emph{estensione} di \(F\).
\end{defn}

\begin{notz}
	Per denotare che \(E\) è un'estensione di \(F\) scriviamo \(E/F\).
\end{notz}

\begin{defn}{Grado dell'estensione}{grado}\index{Grado estensione}
	Il \emph{grado} di un'estensione \(E/F\) è la dimensione di \(E\) come \(F\)-spazio vettoriale:
	\[
		[E:F] := \dim_F E.
	\]
\end{defn}

\begin{notz}
	Diciamo che \(E/F\) è un'estensione \emph{finita} se \([E:F]<+\infty\).
\end{notz}

\begin{ese}
	\begin{itemize}
		\item \(\C/\R\) è un'estensione finita e \([\C:\R]=2\), infatti \(\{1,i\}\) è una \(\R\)-base di \(\C\).
		      \item\(\R/\Q\) è un'estensione infinita. Infatti se fosse \([\R:\Q]<+\infty\), allora esisterebbe \(n\) tale che \(\R\cong_\Q \Q^n\), il che è assurdo in quanto \(\Q\) ha cardinalità numerabile ed \(n-copie\) di \(\Q\) sarebbero ancora numerabili, mentre \(\R\) ha la cardinalità del continuo.
		\item Il campo dei numeri di Gauss \(\Q(i) = \Set{a+i\,b | a,b\in\Q}\) ha dimensione \(2\) come estensione di \(\Q\). Infatti \(\{1,i\}\) è una \(\Q\)-base.
		\item Il campo dei quozienti di un campo \(F\), definito come
		      \[
			      F(X) = \Set{\frac{f}{g} | f,g\in F[X],g\neq 0},
		      \]
		      è un'estensione infinita di \(F\). Infatti \(\Set{1,X,X^2,\ldots,X^n,\ldots}\) è una famiglia infinita in \(F(X)\) che è \(F\)-linearmente indipendente.
	\end{itemize}
\end{ese}

\begin{prop}{Formula del grado}{formulaGrado}
	Consideriamo \(L,E,F\) campi tali che \(L\supset E\supset F\). Allora \(L/F\) ha grado finito se e soltanto se \(L/E\) e \(E/F\) hanno grado finito, nel qual caso vale
	\[
		[L:F]=[L:E][E:F].
	\]
\end{prop}

\begin{proof}
	\graffito{\(\Rightarrow)\)}Supponiamo che \(L/F\) sia finita, allora \(E/F\) è finita poiché \(E\) è un \(F\)-sottospazio di \(L\).
	Inoltre anche \(L/E\) è finita, infatti se \((\a_1,\ldots,\a_r)\) sono generatori di \(L/F\), ovvero
	\[
		L = \Set{a_1\a_1 + \ldots + a_r\a_r | a_i \in F},
	\]
	allora a maggior ragione
	\[
		L = \Set{b_1\a_1 + \ldots + b_r\a_r | b_i \in E}.
	\]
	\graffito{\(\Leftarrow)\)}Supponiamo che \(L/E\) e \(E/F\) siano estensioni finite. Siano \((\a_1,\ldots,\a_t)\) e \((\b_1,\ldots,\b_s)\) rispettivamente una \(F\)-base di \(E\) e una \(E\)-base di \(L\).
	Vogliamo mostrare che
	\[
		(\a_i\b_j)_{\substack{i=1,\ldots,t\\j=1,\ldots,s}},
	\]
	è una \(F\)-base di \(L\). Da ciò seguirebbe \([L:F]=t\,s=[E:F][L:E]\).

	Per prima cosa \((\a_i,\b_j)_{i,j}\) genera \(L\): se \(x\in L\), allora \(x=x_1\b_1+\ldots+x_s\b_s\) con \(x_1,\ldots,x_s\in E\). Ora per ogni \(j=1,\ldots,s\) avremo \(x_j = x_{1j}\a_1+\ldots+x_{t\,j}\a_t\), in quanto \((\a_1,\ldots,\a_t)\) è una base di \(E\). Sostituendo otteniamo
	\[
		x = \sum_{i=1}^t \sum_{j=1}^s x_{i j}\a_i\b_j,
	\]
	ovvero \((\a_i\b_j)_{i,j}\) genera \(L/F\).

	Inoltre \((\a_i,\b_j)_{i,j}\) sono linearmente indipendenti: supponiamo che esistano \(y_{i j}\in F\) tali che
	\[
		\sum_{i,j} y_{i j}\a_i \b_j = 0,
	\]
	allora
	\[
		(\underbrace{y_{1 1}\a_1+\ldots+y_{t 1}\a_t}_{\in E})\,\b_1 + (\underbrace{y_{1 2}\a_1+\ldots+y_{t 2}\a_t}_{\in E})\,\b_2 + \ldots + (\underbrace{y_{1 s}\a_1 + \ldots + y_{t s}\a_t}_{\in E})\,\b_s = 0,
	\]
	da cui \(y_{1 j}\a_1 + \ldots + y_{t j}\a_t = 0\) per ogni \(j\) per la lineare indipendenza di \((\b_1,\ldots,\b_s)\).
	D'altronde poiché \((\a_1,\ldots,\a_t)\) è una \(F\)-base di \(E\) si ha
	\[
		y_{1 j}\a_1 + \ldots + y_{t j}\a_t = 0 \implies y_{1 j},\ldots,y_{t j} = 0,\,\fa j.\qedhere
	\]
\end{proof}
%%%%%%%%%%%%%%%%%%%%%%%%%%%%%%%%%%%%%%
%SOTTOANELLI GENERATI DA SOTTOINSIEMI%
%%%%%%%%%%%%%%%%%%%%%%%%%%%%%%%%%%%%%%
\section{Sottoanelli generati da sottoinsiemi}

\begin{defn}{Sottoanello generato da un sottoinsieme}{sottoanelloGeneratoSottoinsieme}\index{Sottoanello generato!da un sottoinieme}
	Sia \(F\) un sottocampo di un campo \(E\) e sia \(S\subset E\). Definisco il \emph{sottoanello di \(E\) generato da \(S\) su \(F\)} come il più piccolo sottoanello di \(E\) che contiene \(F\) e \(S\):
	\[
		F[S] := \bigcap_{\substack{R\subseteq E \text{ anello}\\F\subseteq R, S\subseteq R}} R.
	\]
\end{defn}

\begin{notz}
	Quando \(S=\{\a_1,\ldots,\a_n\}\) è finito scriviamo \(F[\a_1,\ldots,\a_n]\) per \(F[S]\).
\end{notz}

\begin{pr}
	L'anello \(F[S]\) consiste negli elementi di \(E\) che possono essere scritti come somme finite della forma
	\[
		\sum a_{i_1 \ldots i_n}\a_1^{i_1} \cdot\ldots\cdot \a_n^{i_n}, \qquad\text{con } a_{i_1 \ldots i_n}\in F,\quad \a_i \in S.
	\]
\end{pr}

\begin{proof}
	Supponiamo che \(R\) sia l'insieme di tutti tali elementi. Chiaramente \(R\) è un sottoanello che contiene \(F,S\), per cui \(F[S]\supseteq R\).
	Inoltre ogni altro sottoanello che contiene \(F,S\) deve necessariamente contenere \(R\) per le proprietà di un anello. Quindi \(F[S]=R\).
\end{proof}

\begin{ese}
	L'anello \(\Q[\p]\) consiste in tutti i numeri complessi che possono essere espressi nella forma
	\[
		a_0 + a_1\p + a_2 \p^2 + \ldots + a_n \p^n, \qquad \text{con }a_i\in \Q.
	\]
\end{ese}

\begin{pr}\label{pr:sottocampoAnelloIntegro}
	Sia \(R\) un anello integro e sia \(F\subseteq R\) un campo. Se \(\dim_F R<\infty\) allora \(R\) è un campo.
\end{pr}

\begin{proof}
	Per ogni \(\b\in R,\b\neq 0\) consideriamo la mappa \(f\colon R\to R, x\mapsto \b\,x\). Tale mappa è un applicazione lineari di \(F\)-spazi vettoriali, infatti
	\[
		f(a\,x+b\,y) = \b\,(a\,x+b\,y) = \b\,a\,x+\b\,b\,y = a\,f(x)+b\,f(y).
	\]
	Inoltre \(f\) è iniettiva, infatti \(\Ker f=\Set{x\in R | \b\,x = 0} = (0)\) in quanto \(R\) è integro.
	Ricordando che un endomorfismo fra spazi finiti quando è iniettivo è anche suriettivo si avrà che
	\[
		\ex \a \in R : \b\,\a = 1 \implies \b \text{ invertibile}.\qedhere
	\]
\end{proof}
%%%%%%%%%%%%%%%%%%%%%%%%%%%%%%%%%%%%%%
%SOTTOCAMPI GENERATI DA SOTTOINSIEMI%
%%%%%%%%%%%%%%%%%%%%%%%%%%%%%%%%%%%%%%
\section{Sottocampi generati da sottoinsiemi}

\begin{defn}{Sottocampo generato da un sottoinsieme}{sottocampoGeneratoSottoinsieme}\index{Sottocampo generato!da un sottoinsieme}
	Sia \(F\) un sottocampo di un campo \(E\) e sia \(S\subset E\). Definisco il \emph{sottocampo di \(E\) generato da \(S\) su \(F\)} come il più piccolo sottocampo di \(E\) che contiene \(F\) e \(S\):
	\[
		F(S) := \bigcap_{\substack{L\subseteq E \text{ sottocampo}\\F\subseteq L,S\subseteq L}} L.
	\]
\end{defn}

\begin{notz}
	Quando \(S=\{\a_1,\ldots,\a_n\}\) è finito scriviamo \(F(\a_1,\ldots,\a_n)\) per \(F(S)\).
\end{notz}

\begin{oss}
	Dal momento che i sottocampi sono in particolare sottoanelli, avremo \(F[S]\subseteq F(S)\).
\end{oss}

\begin{pr}
	\(F(S)\) è il campo dei quozienti di \(F[S]\), ovvero
	\[
		f(S) = \Set{\frac{x}{y} | x,y \in F[S], y \neq 0}.
	\]
\end{pr}

\begin{proof}
	Segue dal fatto che \(F(S)\) è un sottocampo che contiene \(F\) e \(S\) e che è contenuto in ogni altro sottocampo con questa proprietà.
\end{proof}

\begin{oss}
	Dalla \autoref{pr:sottocampoAnelloIntegro} sappiamo che \(F[S]\) è un campo se ha dimensione finita su \(F\).
	In tal caso si avrebbe \(F(S)=F[S]\).
\end{oss}

\begin{defn}{Estensione semplice}{estensioneSemplice}\index{Estensione di campi!semplice}
	Un'estensione \(E/F\) si dice \emph{semplice} se esiste \(\a\in E\) tale che \(F(\a)=E\).
\end{defn}

\begin{ese}
	\(\Q(\p)\) e \(\Q[i]\) sono estensioni semplici di \(\Q\).
\end{ese}

\begin{defn}{Estensione finitamente generata}{estensioneFinitamenteGenerata}\index{Estensione di campi!finitamente generata}
	Un'estensione \(E/F\) si dice \emph{finitamente generata} se esistono \(\a_1,\ldots,\a_k\in E\) tale che \(F(\a_1,\ldots,\a_k)=E\).
\end{defn}
%%%%%%%%%%%%%%%%%%
%ANELLI COL GAMBO%
%%%%%%%%%%%%%%%%%%
\section{Anelli col gambo}

Sia \(f(X)\in F[X]\) un polinomio monico di grado \(m\) e sia \((f)\) l'ideale generato da \(f(X)\).
Consideriamo l'anello quoziente \(F[X]/(f)\) e denotiamo con \(\a\) l'immagine di \(X\) in tale anello, ovvero \(\a\) sarà la classe laterale \(X+\big(f(X)\big)\). Ne segue:
\begin{itemize}
	\item La mappa
	      \[
		      F[X] \to F[\a], P(X) \mapsto P(\a),
	      \]
	      è un omomorfismo suriettivo in cui \(f(X)\) viene mappato in \(0\), ovvero \(f(\a)=0\).
	\item Dall'algoritmo euclideo delle divisioni, sappiamo che ogni elemento \(g(X)\in F[X]/(f)\) è rappresentato da un unico polinomio \(r(X)\) con \(\deg r<m\). Quindi ogni elemento di \(F[X]/(f)\) può essere scritto come
	      \begin{equation}\label{eq:campoQuozientePolinomi}
		      a_0+a_1\a_1+\ldots+a_{m-1}\a^{m-1},\qquad\text{con }a_i \in F.\tag{\(\star\)}
	      \end{equation}
	\item Per sommare due elementi nella forma \eqref{eq:campoQuozientePolinomi} è sufficiente sommarne i coefficienti.
	\item Per moltiplicare due elementi nella forma \eqref{eq:campoQuozientePolinomi} si deve moltiplicare nel modo usuale, sfruttando la relazione \(f(\a)=0\) per scrivere i termini di grado superiore a \(m\) in termini di grado inferiore.
	\item Supponiamo che \(f(X)\) sia irriducibile. Allora ogni elemento \(\b\in F[\a]\) ha un inverso. Tale inverso può essere trovato scrivendo \(\b=g(\a)\) con \(g(X)\) un polinomio di grado inferiore a \(m\), per poi applicare l'algoritmo euclideo per ottenere \(a(X)\) e \(b(X)\) tali che
	      \[
		      a(X)f(X)+b(X)g(X) = d(X), \qquad\text{con }d(X) = \big(f(X),g(X)\big).
	      \]
	      Nel nostro caso \(f\) è irriducibile per cui \(d(X)=1\). Inoltre \(\deg g < \deg f\), per cui sostituendo \(\a\) si ottiene
	      \[
		      b(\a)g(\a) = 1 \implies g(\a)^{-1} = b(\a).
	      \]
\end{itemize}

\begin{defn}{Anello col gambo}{anelloGambo}\index{Anello!col gambo}
	Sia \(F\) un campo e sia \(f\) un polinomio monico in \(F[X]\). Si definisce \emph{anello col gambo} l'anello
	\[
		R = \Set{a_0+a_1\a+\ldots+a_{m-1}\a^{m-1} | a_j \in F},
	\]
	con le operazioni definite sopra e tale che
	\[
		F[\a] = R \qquad\text{e}\qquad f(\a) = 0.
	\]
\end{defn}

\begin{oss}
	Se \(f\in F[X]\) è irriducibile \(F[\a]\) è un campo, per cui \(F[\a]=F(\a)\), e inoltre
	\[
		\deg f = n = \big[F[\a]:F\big].
	\]
\end{oss}
%%%%%%%%%%%%%%%%%%%%%%%%%%%%%%%%%%%%%%%%%%%
%
%LEZIONE 05/10/2016 - SECONDA SETTIMANA (3)
%
%%%%%%%%%%%%%%%%%%%%%%%%%%%%%%%%%%%%%%%%%%%
\begin{ese}
	Consideriamo \(f(X) = X^3-3X-1\in \Q[X]\). Avremo che
	\[
		\Q[\a] = \Set{a+b\,\a+c\,\a^2 | a,b,c \in \Q}.
	\]
	Inoltre \(f\) è irriducibile in \(\Q[X]\), per cui \(\Q[\a]\) è un campo ed ha base \((1,\a,\a^2)\) come \(\Q\)-spazio vettoriale.

	Consideriamo adesso \(\b = \a^4+2\a^3+3\in\Q[\a]\). Sapendo che \(\a^3-3\a-1=0\) otteniamo
	\[
		\b = (3\a+1)\,\a + 6\a+2 +3 = 3\a^2+7\a+5.
	\]
	Vogliamo calcolare l'inversa di \(\b\). Dal momento che \(f(X)\) è irriducibile in \(\Q[X]\) segue
	\[
		(X^3-3X-1, 3X^2+7X+5)=1.
	\]
	Applicando l'algoritmo di Euclide otteniamo l'identità di Bezout
	\[
		(X^3-3X-1)\left( -\frac{7}{37}X+\frac{29}{111} \right)+(3X^2+7X+5)\left( \frac{7}{111}X^2+\frac{26}{111}X+\frac{28}{111} \right)=1,
	\]
	da cui segue immediatamente
	\[
		\b^{-1} = \frac{7}{111}\a^2+\frac{26}{111}\a+\frac{28}{111}.
	\]
\end{ese}
%%%%%%%%%%%%%%%%%%%%%%%%%%%%%%%%%%%
%ELEMENTI ALGEBRICI E TRASCENDENTI%
%%%%%%%%%%%%%%%%%%%%%%%%%%%%%%%%%%%
\section{Elementi algebrici e trascendenti}

Sia \(E/F\) un'estensione e sia \(\a\in E\), avremo che
\[
	\j\colon F[X] \to E, h(X) \mapsto h(\a),
\]
è un omomorfismo di anelli.

\begin{defn}{Elemento trascendente}{elementoTrascendente}\index{Elemento trascendente}
	Se \(\Ker \j=(0)\) diciamo che \(\a\) è \emph{trascendente} su \(F\).
\end{defn}

\begin{oss}
	Quando un elemento è trascendente significa che per ogni \(h\in F[X],h\neq 0\) si ha \(h(\a)\neq 0\).
	Ciò significa che l'immagine di \(\j\) è isomorfa a \(F[X]\), ovvero \(F[\a]\cong F[X]\). Inoltre avremo
	\[
		\begin{tikzcd}
			F[X] \arrow[dr, hook] \arrow[rr, hook] &                       & E\\
			& F(X) \arrow[ur, hook] &
		\end{tikzcd}
		\qquad \text{con }F(X) \hookrightarrow E, \frac{h_1(X)}{h_2(X)} \mapsto \frac{h_1(\a)}{h_2(\a)}.
	\]
\end{oss}

\begin{ese}
	Prendiamo 	\(E=\C,F=\Q\) e \(\a=\p\). Siccome \(\p\) è trascendente si ha
	\[
		\Q[X] \cong \Q[\p] \subseteq \C.
	\]
	Ciò significa che \(\Q[\p]\) è algebricamente indistinguibile da \(\Q[X]\).
\end{ese}

\begin{defn}{Elemento algebrico}{elementoAlgebrico}\index{Elemento algebrico}
	Se \(\Ker \j=(f_\a)\) diciamo che \(\a\) è \emph{algebrico} su \(F\).
\end{defn}

\begin{defn}{Polinomio minimo}{polinomioMinimo}\index{Polinomio minimo}
	I polinomi \(g\) tali che \(g(\a)=0\) formano un ideale non banale in \(F[X]\). Tale ideale è generato dal più piccolo polinomio monico \(f_\a\) tale che \(f_\a(\a)=0\). Definiamo \(f_\a\) come il \emph{polinomio minimo di \(\a\) su \(F\)}.
\end{defn}

\begin{oss}
	Il polinomio minimo è irriducibile poiché altrimenti vi sarebbero due elementi non nulli di \(E\) che hanno come prodotto zero.
\end{oss}

\begin{oss}
	Siccome \(F[X]/(f_\a)\cong F[\a]\), in \(F[\a]\) possiamo assumere che tutte le espressioni polinomiali abbiano grado minore di \(\deg f_\a\).
	Cioè l'immagine tramite \(\j\) è il campo col gambo \(F[\a]\) di gambo \(f_\a\). In questo caso si ha che
	\[
		F[\a] = F(\a).
	\]
	Inoltre \(\big[F[\a]:F\big]=\deg f_\a\) e \((1,\a,\ldots,\a^{\deg f_\a-1})\) è una base.
\end{oss}

\begin{prop}{Caratterizzazione del polinomio minimo}{caratterizzazionePolinomioMinimo}
	Se \(E/F\) è un'estensione e \(\a\in E\) è algebrico su \(F\), il polinomio minimo \(f_\a\) è caratterizzato come polinomio di \(F[X]\) da ognuna seguenti condizioni:
	\begin{itemize}
		\item L'unico polinomio monico e irriducibile in \(F[X]\) che si annulla in \(\a\).
		\item L'unico polinomio monico con la proprietà che se \(g(X)\in F[X]\) e \(g(\a)=0\) allora \(f_\a\mid g\).
		\item L'unico polinomio monico che si annulla in \(\a\) e ha grado minimo.
	\end{itemize}
\end{prop}

\begin{ese}
	Su \(\big(\Q[\a],\a^3=3\a+1\big)\) prendiamo \(\a^2\in\Q[\a]\).
	Vogliamo stabilire se \(\a^2\) è trascendete su \(\Q\). Se non lo è vogliamo trovare il suo polinomio minimo.

	Prendiamo un generico polinomio \(X^3+A\,X^2+B\,X+C\). Se trovo \(A,B,C\in \Q\) tali che una volta sostituito \(\a^2\) nel polinomio ottengo zero, ho trovato il polinomio minimo:
	\[
		\begin{split}
			\a^6+A\,\a^4+B\,\a^2+C = 0 & \iff (3\a+1)^2+A\,(3\a^2+\a)+B\,\a^2+C = 0\\
			& \iff (9\a^2+6\a+1)+A\,(3\a^2+\a)+B\,\a^2+C= 0\\
			& \iff (3A+B+9)\a^2+(A+6)\a+(C+1)=0,
		\end{split}
	\]
	da cui
	\[
		\begin{cases}
			3A+B+9 = 0 \\
			A+6 = 0    \\
			C+1=0
		\end{cases}
		\implies
		\begin{cases}
			A=-6 \\
			B=9  \\
			C=-1
		\end{cases}
	\]
	ovvero \(f_{\a^2}(X)=X^3-6X^2+9X-1\). Quindi
	\[
		\Q\subseteq \Q(\a^2) \subseteq \Q[\a] \cong \Q(\a).
	\]
	Con \(\big[\Q[\a]:\Q\big]=\deg(\a^3-3\a-1)=3\) e ancora \(\big[\Q(\a^2):\Q\big]=3\), da cui
	\[
		\big[\Q[\a]:\Q(\a^2)\big] = 1 \implies \Q(\a) = \Q(\a^2).
	\]
\end{ese}

\begin{defn}{Estensione algebrica}{estensioneAlgebrica}\index{Estensione di campi!algebrica}
	Un'estensione \(E/F\) si dice \emph{algebrica} se ogni elemento \(\a\in E\) è algebrico in \(F\).
\end{defn}

\begin{defn}{Estensione trascendente}{estensioneTrascendente}\index{Estensione di campi!trascendente}
	Un'estensione \(E/F\) si dice \emph{trascendente} se non è algebrica, ovvero se esiste un elemento \(\b\in E\) che è trascendente su \(F\).
\end{defn}

\begin{prop}{Caratterizzazione estensione finita}{caratterizzazioneEstensioneFinita}
	Sia \(E/F\) un'estensione.
	Allora \(E/F\) è finita se e soltanto se \(E/F\) e algebrica e finitamente generata.
\end{prop}

\begin{proof}
	\graffito{\(\Rightarrow)\)}Supponiamo che \(E/F\) sia un'estensione finita, allora
	\begin{itemize}
		\item \(E/F\) algebrica: se per assurdo esistesse \(\b\in E\) trascendente su \(F\), si avrebbe che \((1,\b,\ldots,\b^n,\ldots)\) sarebbero linearmente indipendenti su \(F\). Ciò implicherebbe che \(\dim_F E=\infty\) che è assurdo per la finitezza dell'estensione.
		\item \(E/F\) finitamente generate: se \(E=F\) allora \(E=F(1)\); se invece \(E\supset F\), preso \(\a_1\in E-F\) avremmo
		      \[
			      E \supseteq F[\a_1] \supset F.
		      \]
		      In particolare \(F[\a_1]/F\) è finita per cui \(F[\a_1]=F(\a_1)\).
		      Ora se \(E=F[\a_1]\) abbiamo mostrato la tesi, altrimenti se \(E\supset F[\a_1]\) possiamo trovare \(\a_2\in E-F[\a_1]\) ottenendo
		      \[
			      E\supset F[\a_1,\a_2]\supset F[\a_1]\supset F,
		      \]
		      che sono tutte estensioni finite. Posso quindi iterare il processo scrivendo \(E\supseteq F[\a_1,\ldots,\a_k]=F(\a_1,\ldots,\a_k)\). Tale processo deve terminare poiché
		      \[
			      n_1 n_2 \cdot\ldots\cdot n_k = \big[F[\a_1,\ldots,\a_k]:F\big] = \big[F[\a_1,\ldots,\a_k]:F[\a_1,\ldots,\a_{k-1}]\big]\big[F[\a_1,\ldots,\a_{k-1}]:F\big],
		      \]
		      dove \(n_j=\big[F[\a_1,\ldots,\a_j]:F[\a_1,\ldots,\a_{j-1}]\big]>1\). D'altronde \(n_1 \cdot\ldots\cdot n_k \mid [E:F]<\infty\), per cui deve esistere \(k_0\) tale che
		      \[
			      n_1 \cdot\ldots\cdot n_{k_0} = [E:F] \implies E = F[\a_1,\ldots,\a_{k_0}].
		      \]
	\end{itemize}
	\graffito{\(\Leftarrow)\)}Supponiamo che \(E/F\) sia un'estensione finitamente generata, con \(E=F[\a_1,\ldots,\a_k]\), e algebrica. Dobbiamo mostrare che è finita.
	\begin{itemize}
		\item Se \(k=1\) allora \(E=F[\a]/F\) è finita di grado \(\deg f_\a\) poichè \(F[\a]\) risulta essere un campo col gambo.
		\item Se \(k>1\) possiamo scrivere \(E=F[\a_1,\ldots,\a_{k-1}][\a_k]\) che per induzione ci dice che \(F[\a_1,\ldots,\a_{k-1}]/F\) è finita.
		      Inoltre \(E/F[\a_1,\ldots,\a_{k-1}]\) è finita perché è un campo col gambo, il cui gambo è il polinomio minimo \(f_{\a_k}\) su \(F[\a_1,\ldots,\a_{k-1}][\a_k]\) e il suo grado è
		      \[
			      [E:F] = \underbrace{\big[E:F[\a_1,\ldots,\a_{k-1}]\big]}_{<\infty}\underbrace{\big[F[\a_1,\ldots,\a_{k-1}]:F\big]}_{<\infty} < \infty.\qedhere
		      \]
	\end{itemize}
\end{proof}
%%%%%%%%%%%%%%%%%%%%%%%%%%%%%%%%%%%%%%%%%
%
%LEZIONE 11/10/2016 - TERZA SETTIMANA (1)
%
%%%%%%%%%%%%%%%%%%%%%%%%%%%%%%%%%%%%%%%%%
\begin{cor}
	Siano \(E/F\) un'estensione algebrica e \(R\) un anello tale che \(F\subset R\subset E\).
	Allora \(R\) è un campo.
\end{cor}

\begin{proof}
	Preso \(\a\in R\setminus\{0\}\) avremo che \(\a\) è algebrico su \(F\) in quanto elemento di \(E\).
	Ora \(F[\a]\) è algebrico e finitamente generato, quindi per la proposizione precedente \(F[\a]\) è finito, ovvero \(F[\a]=F(\a)\).
	Da cui segue
	\[
		\frac{1}{\a} \in F[\a] \subset R,
	\]
	poiché ciò vale per ogni elemento non nullo di \(R\), segue che \(R\) è un campo.
\end{proof}

\begin{cor}
	Siano \(E/F\) e \(L/E\) due estensioni algebriche. Allora \(L/F\) è un'estensione algebrica.
\end{cor}

\begin{proof}
	Sia \(\a\in L\). Dal momento che \(L/E\) è algebrica esisterà \(f\in E[X]\) monico tale che \(f(\a)=0\). Dove
	\[
		f(X) = X^m + a_1 X^{m-1}+\ldots+a_m, \qquad\text{con }a_j\in E.
	\]
	Per ipotesi \(E/F\) è algebrica, per cui \(a_j\in E \implies F[a_1,\ldots,a_m]\) è ancora algebrica su \(F\), inoltre è finitamente generata. Quindi per la proposizione \(F[a_1,\ldots,a_m]/F\) è finita. Inoltre anche
	\[
		F[a_1,\ldots,a_m,\a]/F[a_1,\ldots,a_m],
	\]
	è finita poiché \(F[a_1,\ldots,a_m,\a]=F[a_1,\ldots,a_m][\a]\) è l'anello col gambo su \(F[a_1,\ldots,a_m]\).
	Quindi per la \hyperref[pr:formulaGrado]{la formula del grado} avremo
	\[
		\big[F[a_1,\ldots,a_m,\a]:F\big] = \big[F[a_1,\ldots,a_m,\a]:F[a_1,\ldots,a_m]\big]\big[F[a_1,\ldots,a_m]:F\big] < +\infty.
	\]
	Ovvero \(F[a_1,\ldots,a_m,\a]\) è finito, e quindi algebrico, su \(F\).
\end{proof}
%%%%%%%%%%%%%%%%%%%%%%%%%%%%%%%%%%%%%%%%%%
%
%LEZIONE 17/10/2016 - QUARTA SETTIMANA (1)
%
%%%%%%%%%%%%%%%%%%%%%%%%%%%%%%%%%%%%%%%%%%
%%%%%%%%%%%%%%%%%%%%%
%NUMERI TRASCENDENTI%
%%%%%%%%%%%%%%%%%%%%%
\section{Numeri trascendenti}

Un numero complesso si dice \emph{algebrico} o \emph{trascendente} secondo che sia algebrico o trascendente su \(\Q\).
Per comodità definiamo l'insieme dei numeri algebrici
\[
	\mathbb{A} = \Set{\a \in \C | \a \text{ è algebrico su }\Q}.
\]
Si può dimostrare che \(\mathbb{A}\) è un campo e che ha cardinalità numerabile.

Ora riportiamo alcuni cenni storici:
\begin{description}
	\item[1844] Liouville dimostra l'esistenza di numeri trascendenti, ovvero che \(\C-\mathbb{A}\neq \emptyset\).
	\item[1873] Hermite dimostra che \(e\) è un numero trascendente.
	\item[1874] Cantor dimosta che l'insieme \(\mathbb{A}\) è numerabile e che \(\R\) non lo è. Ciò prova che la maggior parte dei reali sono trascendenti, anche se è molto difficile dimostrare che un numero specifico lo sia.
	\item[1882] Lindemann dimostra che \(\p\) è trascendente.
	\item[1934] Gel'fond dimostra che se \(\a\) e \(\b\) sono algebrici con \(\a,\b\neq 0,1\) e \(\b\not\in \Q\), allora \(\a^\b\) è trascendente.
	\item[2016] Non è stato ancora dimostrato se la costante di Eulero-Mascheroni
		\[
			\g = \lim_{N \to +\infty} \bigg(\sum_{n=1}^N \frac{1}{n}-\ln N\bigg),
		\]
		è trascendente o addirittura se è irrazionale.
	\item[2016] Nonostante i numeri \(\p+e\) e \(\p-e\) siano certamente trascendenti non è stato ancora dimostrato se sono irrazionali.
\end{description}

\begin{prop}{Numerabilità dell'insieme dei numeri algebrici}{numerabilitàAlgebrici}
	L'insieme \(\mathbb{A}\) dei numeri algebrici è numerabile.
\end{prop}

\begin{proof}
	Definiamo l'altezza \(H(r)\) di un razionale \(r=n/m\), con \((n,m)=1\) e \(n\in\Z,m\in\N\), come
	\[
		H(r) = \max{\abs{n},m}.
	\]
	\`E è facile convincersi che vi sono solo un numero finito di razionali con la proprietà di avere l'altezza minore di un certo \(N\) fissato.
	Definiamo inoltre l'altezza di un polinomio monico come
	\[
		H(a_0+a_1 X+\ldots+a_{n-1}X^{n-1}+X^n) = \max\Set{H(a_0),\ldots,H(a_{n-1})}.
	\]
	La strategia è mostrare che, definito \(B_n=\Set{\a \in \mathbb{A} | \deg f_\a \le n, H(f_\a)\le n}\), si abbia
	\[
		\mathbb{A} = \bigcup_{n\in\N} B_n \qquad\text{e}\qquad \#B_n < +\infty.
	\]
	Da cui seguirebbe che \(\# \mathbb{A}\) è numerabile in quanto unione numerabile di insiemi finiti.
	Anche in questo caso è facile convincersi che \(\#B_n\) è finito.
\end{proof}

\begin{teor}{Trascendenza di un numero di Liouville}{trascendenzaNumeroLiuoville}
	Il seguente numero di Liouville
	\[
		\a = \sum_{n=0}^\infty \frac{1}{2^{n!}}
	\]
	è trascendente.
\end{teor}

\begin{proof}
	Supponiamo per assurdo che \(\a\) sia algebrico. Scriviamo il polinomio minimo di \(\a\):
	\[
		f_\a(X) = X^d+a_1 X^{d-1}+\ldots+ a_d, \qquad\text{con }a_j\in \Q.
	\]
	Fissato \(N\in \N\) definiamo
	\[
		\Sigma_N = \sum_{n=0}^N \frac{1}{2^{n!}}.
	\]
	Chiaramente avremo che \(\Sigma_N\in \Q\) e \(\Sigma_N \to \a\) monotonicamente.
	Inoltre avremo che \(X_N = f_\a(\Sigma_N)\in\Q-\{0\}\), in quanto \(f_\a\) è un polinomio irriducibile in \(\Q\) e pertanto non può avere radici razionali a meno che non sia di grado uno, ma in tal caso la sua unica radice sarebbe \(\a\).

	Sia \(D\in \Z\) tale che \(D\,f_\a(X)\in\Z[X]\), per cui avremo
	\[
		{(2^{N!})}^d D\,X_n \in \Z-\{0\} \qquad\text{e}\qquad 1 \le \abs{{(2^{N!})}^d D\,X_N}.
	\]
	Ora per il Teorema Fondamentale dell'Algebra
	\[
		f_\a(X) = (X-\a)(X-\a_2)\cdot\ldots\cdot(X-\a_d),
	\]
	da cui
	\[
		\abs{X_N} = \abs{\Sigma_N-\a} \prod_{j=2}^d \abs{\Sigma_N-\a_j}.
	\]
	In particolare, da \(k\ge N+1 \implies k!-(N+1)!\ge k\), avremo
	\[
		\abs{\Sigma_N-\a} = \sum_{k=N+1}^\infty \frac{1}{2^{k!}} \le \frac{1}{2^{(N+1)!}}\sum_{k=0}^\infty \frac{1}{2^k} \le \frac{2}{2^{(N+1)!}}
	\]
	Inoltre
	\[
		\abs{\Sigma_N-\a_j} \le \Sigma_N + \abs{\a_j} \le \a + M, \qquad\text{con }M = \max\Set{\abs{\a_2},\ldots,\abs{\a_d}}.
	\]
	Quindi
	\[
		\abs{X_N} \le \frac{2}{2^{(N+1)!}}\,(\a+M)^{d-1},
	\]
	ovvero
	\[
		1 \le \abs{{(2^{N!})}^d D\,X_N} \le \frac{{(2^{N!})}^d\,2}{2^{(N+1)!}}\,(\a+M)^{d-1} = \left( \frac{2^d}{2^{N+1}} \right)^{N!}\,2(\a+M)^{d-1},
	\]
	che tende a zero, da cui l'assurdo.
\end{proof}
%%%%%%%%%%%%%%%%%%%%%%%%%%%%%
%CAMPI ALGEBRICAMENTE CHIUSI%
%%%%%%%%%%%%%%%%%%%%%%%%%%%%%
\section{Campi algebricamente chiusi}

\begin{notz}
	Diciamo che un polinomio si \emph{spezza} su un campo \(F\) se può essere scritto come prodotto di polinomi di grado \(1\) in \(F[X]\).
\end{notz}

\begin{defn}{Campo algebricamente chiuso}{campoAlgebricamenteChiuso}\index{Campo!algebricamente chiuso}
	Un campo \(\Omega\) si dice \emph{algebricamente chiuso} se ogni polinomio non costante in \(\Omega[X]\) si spezza in \(\Omega\).
\end{defn}

\begin{oss}
	\(\C\) è algebricamente chiuso come immediata conseguenza del Teorema Fondamentale dell'Algebra.
\end{oss}

\begin{prop}{Caratterizzazione di campi algebricamente chiusi}{caratterizzazioneCampiAlgebricamenteChiusi}
	Sia \(\Omega\) un campo, allora le seguenti affermazioni sono equivalenti:
	\begin{enumerate}
		\item Ogni polinomio non costante in \(\Omega[X]\) si spezza in \(\Omega\).
		\item Ogni polinomio non costante in \(\Omega\) ha almeno una radice in \(\Omega\).
		\item Se un polinomio su \(\Omega[X]\) è irriducibile allora ha grado \(1\).
		\item Se \(E/\Omega\) è un'estensione finita allora \(E=\Omega\).
	\end{enumerate}
\end{prop}

\begin{proof}
	Le implicazioni \((1)\implies (2)\implies (3)\implies (1)\) sono ovvie.

	\graffito{\((3)\implies (4)\)} Sia \(E/\Omega\) un'estensione finita e sia \(\a\in E\).
	Il polinomio minimo \(f_\a(X)\in\Omega[X]\) è irriducibile, quindi, per ipotesi, \(\deg f_\a = 1\).
	In particolare
	\[
		f_\a(X) = X-\a \implies \a \in \Omega,
	\]
	da cui \(E\subseteq \Omega\).

	\graffito{\((4)\implies (3)\)}Sia \(f(X)\in\Omega[X]\) un polinomio irriducibile.
	Consideriamo \(\Omega'=\Omega[X]/(f)\), avremo che \(\Omega'/\Omega\) è un'estensione finita con
	\[
		[\Omega':\Omega] = \deg f.
	\]
	D'altronde per ipotesi \(\Omega'=\Omega\), quindi \(1=[\Omega':\Omega]=\deg g\).
\end{proof}

\begin{oss}
	\(\C(X)/\C\) è un'estensione non banale di \(\C\) ma è infinita.
\end{oss}

\begin{defn}{Chiusura algebrica}{chiusuraAlgebrica}\index{Chiusura!algebrica}
	Un'estensione \(\Omega/F\) si definisce \emph{chiusura algebrica} di \(F\) se
	\begin{itemize}
		\item \(\Omega/F\) è un'estensione algebrica.
		\item \(\Omega\) è algebricamente chiuso.
	\end{itemize}
\end{defn}

\begin{ese}
	\(\C/\R\) è una chiusura algebrica, d'altronde \(\C/\Q\) non lo è in quanto non è un'estensione algebrica.
\end{ese}

\begin{pr}
	Sia \(\Omega/F\) è un'estensione algebrica e supponiamo che per ogni \(f\in F[X]\) si abbia che \(f\) si spezza in \(\Omega[X]\).
	Allora \(\Omega\) è algebricamente chiuso.
\end{pr}

\begin{proof}
	Sia \(f\in \Omega[X]\) con \(\deg f\ge 1\). Vogliamo trovare una radice di \(f\) in \(\Omega\).
	Scriviamo il polinomio come
	\[
		f(X) = a_n X^n + \ldots + a_1 X + a_0, \qquad\text{con }a_j\in \Omega.
	\]
	Supponiamo per assurdo che \(f\) sia irriducibile. Allora possiamo trovare un'estensione \(L/\Omega\) tale che \(f\) ha una radice in \(L\).
	Infatti è sufficiente prendere \(L=\Omega[X]/(f)\).
	In particolare avremo \(L/\Omega\) finita e \(f(\a)=0\) per una certa \(\a\in L\). Inoltre
	\[
		F \subseteq F[a_0,\ldots,a_n] \subseteq F[a_0,\ldots,a_n,\a] \subseteq L = \Omega[\a].
	\]
	Dove la prima estensione è finita in quanto algebrica e finitamente generata, mentre la seconda lo è poiché costruita con la radice di un polinomio nel campo di partenza. Quindi per la \hyperref[pr:formulaGrado]{la formula del grado} avremo che \(F[a_0,\ldots,a_n,\a]/F\) è finita, e in particolare \(\a\) soddisfa un polinomio irriducibile in \(F[X]\).
	Ovvero esiste \(g\in F[X]\) tale che \(g(\a)=0\).
	Ora, per ipotesi, \(g\) si spezza in \(\Omega\), per cui \(\a \in \Omega\).
\end{proof}

\begin{oss}
	In particolare \(\Omega\) è una chiusura algebrica di \(F\).
\end{oss}

\begin{pr}
	Sia \(E/F\) un'estensione, allora
	\[
		\mathbb{A}_{E/F} = \Set{\a \in E | \a \text{ è algebrico su }F}
	\]
	è un campo.
\end{pr}

\begin{proof}
	Siano \(\a,\b\) elementi algebrici di \(F\). Allora \(F[\a,\b]\) è un campo ed un'estensione finita di \(F\), in quanto \hyperref[pr:caratterizzazioneEstensioneFinita]{algebrico e finitamente generato}.
	In particolare ogni elemento di \(F[\a,\b]\) sarà algebrico su \(F\), compresi
	\[
		\a+\b; \qquad \a\,\b; \qquad \frac{\a}{\b}.\qedhere
	\]
\end{proof}